%!TEX root = main.tex

\section{Intuitions}
\label{sec:intuitions}
Reinforcement learning aims at tackling a wide variety of real world problems.
Among them obviously stand physical problems, where an agent interacts with a
physical environment, e.g.\ the real world.

Resistance to a change of time discretization is a desirable property for
algorithms aiming at attacking the reinforcement learning field in physical
setups: such algorithms should only improve as the time discretization is
refined, as refining the time discretization can only increase the amount of 
eploitable information.

However, some reinforcement learning algorithms such as \emph{Q-Learning} are
ill-behaved in the regime of small time discretization. In what follow, we
analyze in an handwavy manner the reason for this failure, and give an alternative
algorithm that remains viable in the limit of small discretizations.

\subsection{Notations}
In what follows, we loosely consider the notion of a $\emph{Markov Decision Process}$
$\mathcal{M}$ with discretization step $\deltat$. Such an MDP is the discretization
of a continuous MDP with discretization step $\deltat$. Notably, this means that
contiguous states are at distance $\bigO{\deltat}$ and that instantaneous rewards
have magnitude $\bigO{\deltat}$ for any policy, i.e.
\begin{align}
	\|s_{t + \deltat} - s_t\| &= \bigO{\deltat}\\
	r_t = \reward_t \deltat, &\quad \reward_t = \bigO{1}.
\end{align}
Policies are functions from the state space $\statespace$ to distributions on the
$\actionspace$
\begin{equation}
	\pi\colon \statespace \mapsto \mathcal{D}\left(\actionspace\right).
\end{equation}
The probability of an action $a$ in a given state $s$ under policy $\pi$ is
denoted by $\pi\left(a\mid s\right)$.

We consider discount factors of the form
\begin{equation}
	\gamma' = \left[1 - (1 - \gamma) \deltat\right],
\end{equation}
so has to have an effective discount factor close to $\gamma$.

The state value function and action-state value function of a policy
$\pi$ are defined as
\begin{align}
	V^\pi(s) &= \BE_\pi\left[
		\sum\limits_{t=s}^\infty
		{\gamma'}^{s - t} \reward_{t + (s-t)\deltat} \deltat
		\mid
		s_t = s
	\right]\\
	Q^\pi(s, a) &= \BE_\pi\left[
		\sum\limits_{t=s}^\infty
		{\gamma'}^{s - t} \reward_s \deltat
		\mid
		s_t = s, a_t = a
	\right].
\end{align}
Both functions satisfy recurrent \emph{Bellman equations}
\begin{align}
	V^\pi(s) &= \BE_\pi\left[
		\reward_t \deltat + \gamma' V^\pi(s_{t + \deltat})
		\mid
	s_t = s\right]\\
	Q^\pi(s, a) &= \BE_\pi\left[
		\reward_t \deltat + \gamma' V^\pi(s_{t + \deltat})
		\mid
	s_t = s, a_t = a\right].
\end{align}

\subsection{$Q^\pi = V^\pi$ in continuous time}
\emph{Q-learning} uses the state-action value function to determine which
action optimizes the future return.

However, in continuous time, the state-action value function no longer carries
any information on which action yields the best return. Intuitively, as the
discretization step goes to zero, the difference in resulting state between two
actions is of magnitude $\bigO{\deltat}$. Consequently, the value functions of
the corresponding states, and thus the state-action value functions only differ
by a factor of magnitude $\bigO{\deltat}$.

Formally,
\begin{align}
	V^\pi(s) - Q^\pi(s, a) &= 
	(\BE_\pi\left[\reward_t \deltat \mid s_t=s\right] -
	\BE_\pi\left[\reward_t \deltat \mid s_t=s\right]) \deltat +\\
	&\gamma'(\BE_\pi\left[
		V^\pi(s_{t + \deltat})
		\mid s_t = s
	\right] - \BE_\pi\left[
		V^\pi(s_{t + \deltat})
		\mid s_t = s, a_t=a
	\right]).\nonumber
\end{align}
Since $s_{t+\deltat} = s_t + \bigO{\deltat}$ and $V^\pi$ is differentiable,
$V^\pi(s_{t+\deltat}) = V^\pi(s_t) + \bigO{\deltat}$. Consequently,
\begin{equation}
	V^\pi(s) - Q^\pi(s, a) = \bigO{\deltat}.
\end{equation}

Notably, decision based on the values or relative values of an approximation of
the $Q^\pi(s, a)$'s for different $a$ are only meaningful when the
approximation is tighter than $\bigO{\deltat}$, at least in term of relative
values, i.e.  
\begin{equation*}
	|\tilde{Q}^\pi(s, a) - \tilde{Q}^\pi(s, a') - Q^\pi(s, a) +
	Q^\pi(s, a')| = \bigO{\deltat}.
\end{equation*}.

Approximating with this degree of precision becomes harder as $\bigO{\deltat}$
decreases. \NDC{This is not convincing at the moment: as $\deltat$ decreases,
	the number of observations made increases, and this may help get a better
	estimate $\tilde{Q}^\pi(s, a)$. My guess is that the number of observations
	required to obtain an approximation of order $\varepsilon$ is superlinear in
	$\frac{1}{\varepsilon}$ and thus the previous argument holds. We either need
a simple theoretical setup where this works or an experimental one.}
