\documentclass[11pt,a4paper]{article}
%\usepackage[colorlinks=true,linkcolor=blue]{hyperref} --clashes with beamer...
\usepackage{cmap}
\usepackage[utf8x]{inputenc}
\usepackage[T1]{fontenc}
\usepackage[english]{babel}
\usepackage{amssymb,amsmath}
\usepackage{amsthm}
\usepackage{bbm}
\usepackage{microtype}
\usepackage{lmodern}

%To get bold small caps with lmodern
{
\rmfamily
\DeclareFontShape{T1}{lmr}{m}{sc}{<->ssub*cmr/m/sc}{}
\DeclareFontShape{T1}{lmr}{b}{sc}{<->ssub*cmr/b/sc}{}
\DeclareFontShape{T1}{lmr}{bx}{sc}{<->ssub*cmr/bx/sc}{}
}

% \newif\ifpdf
% \ifx\pdfoutput\undefined
%   \pdffalse
% \else
%   \pdfoutput=1
%   \pdftrue
% \fi
% 
% \ifpdf
%   \usepackage{aeguill}
%   \pdfcompresslevel9
% \else
% \fi
% 

%\usepackage{lmodern}

\newcommand{\augmentehauteur}[1]{
\addtolength{\topmargin}{-#1}
\addtolength{\textheight}{#1}
\addtolength{\textheight}{#1}
}

\newcommand{\augmentelargeur}[1]{
\addtolength{\evensidemargin}{-#1}
\addtolength{\oddsidemargin}{-#1}
\addtolength{\textwidth}{#1}
\addtolength{\textwidth}{#1}
}

\newcommand{\thmheadercommand}[1]{\textbf{\scshape{}#1.\\*}}

\newtheoremstyle{yannthm}{\topsep}{\topsep}{\slshape}{}{\scshape\bfseries}{.}{.5em}{%
\thmname{#1}\thmnumber{ #2}\thmnote{#3}%
}
\newtheoremstyle{yannthm2}{\topsep}{\topsep}{}{}{\scshape\bfseries}{.}{.5em}{%
\thmname{#1}\thmnumber{ #2}\thmnote{#3}%
}


\newcommand{\mynumero}{n°}


%\def\d{{\mathrm{d}}}
%\def\d{\operatorname{d}\!}
\def\d{\operatorname{d}\!{}}
%\def\d{\operatorname{}\!\mathrm{d}}

\def\N{{\mathbb{N}}}
\def\Z{{\mathbb{Z}}}
\def\Nstar{{\mathbb{N}^\star}}
\def\Q{{\mathbb{Q}}}
\def\R{{\mathbb{R}}}
\def\C{{\mathbb{C}}}

\renewcommand{\geq}{\geqslant}
\renewcommand{\leq}{\leqslant}

\renewcommand{\emptyset}{\varnothing}

\newcommand{\deq}{\mathrel{\mathop{:}}=}
\newcommand{\eqd}{=\mathrel{\mathop{:}}}

\newcommand{\from}{\colon} % correct ':' in f\from X \to Y
\newcommand{\st}{\mid} % set builder
%\newcommand{\st}{\mathrel{}\middle|\mathrel{}} % set builder

\def\eps{\varepsilon}
\renewcommand{\epsilon}{\varepsilon}
\renewcommand{\phi}{\varphi}

\def\ds{\displaystyle}

\DeclareMathOperator{\dist}{dist}
\DeclareMathOperator{\diam}{diam}
\DeclareMathOperator{\vol}{vol}
\DeclareMathOperator{\Ric}{Ric}

\DeclareMathOperator{\lap}{\Delta\!}
\DeclareMathOperator{\nab}{\nabla\!\!}
\DeclareMathOperator{\Hess}{Hess}

\DeclareMathOperator{\Ent}{Ent}
\DeclareMathOperator{\Var}{Var}
\DeclareMathOperator{\Cov}{Cov}
\let\oldPr\Pr
\renewcommand{\Pr}{\oldPr\nolimits}
\newcommand{\E}{\mathbb{E}}
\newcommand{\KL}[2]{\mathrm{KL}\!\left(#1 \,|\hspace{-.15ex}|\,#2\right)}

\DeclareMathOperator{\mult}{mult}
\DeclareMathOperator{\Card}{Card}
\DeclareMathOperator{\Aut}{Aut}
\DeclareMathOperator{\Epi}{Epi}
\DeclareMathOperator{\Spec}{Sp}
\DeclareMathOperator{\Ker}{Ker}
\DeclareMathOperator{\Img}{Im}
\DeclareMathOperator{\Tr}{Tr}
\DeclareMathOperator{\tr}{tr}
\DeclareMathOperator{\Tor}{Tor}
\DeclareMathOperator{\Ext}{Ext}
\DeclareMathOperator{\Hom}{Hom}
\DeclareMathOperator{\End}{End}
\DeclareMathOperator{\coker}{coker}
\DeclareMathOperator{\Id}{Id}
\DeclareMathOperator{\id}{id}
\DeclareMathOperator{\diag}{diag}


\newcommand{\abs}[1]{\left\lvert#1\right\rvert}
\newcommand{\norm}[1]{\left\lVert#1\right\rVert}
\newcommand{\scal}[2]{\left< \, #1 \mid #2 \, \right>}
\newcommand{\1}{\mathbbm{1}}

\newcommand{\ilim}[1]{\underset{#1}{\underrightarrow{\lim\vspace{.5ex}}}\,}
\newcommand{\plim}[1]{\underset{#1}{\underleftarrow{\lim\vspace{.5ex}}}\,}

\DeclareMathOperator*{\vlimsup}{\varlimsup}
\DeclareMathOperator*{\vliminf}{\varliminf}

\newcommand{\presgroup}[2]{\left\langle\,#1 \mid  #2\,\right\rangle}

\newcommand{\twopi}{2\hspace{-.23em}\pi}

\DeclareMathOperator*{\argmax}{arg\,max}
\DeclareMathOperator*{\argmin}{arg\,min}



%\newenvironment{dem}[1][]{\smallskip\noindent{\thmheadercommand{Proof#1}} \ignorespaces}{\hfill$\square$\medskip}

\newenvironment{dem}[1][]{\begin{proof}[\thmheadercommand{Proof#1}]~\newline\ignorespaces}{\end{proof}}

{
\theoremstyle{yannthm}
\newtheorem{defi}{Definition}
\newtheorem*{defi*}{Definition}
\newtheorem{prop}[defi]{Proposition}
\newtheorem*{prop*}{Proposition}
\newtheorem{thm}[defi]{Theorem}
\newtheorem*{thm*}{Theorem}
\newtheorem{lem}[defi]{Lemma}
\newtheorem*{lem*}{Lemma}
\newtheorem{cor}[defi]{Corollary}
\newtheorem*{cor*}{Corollary}
\newtheorem{ex}[defi]{Example}
\newtheorem*{ex*}{Example}
\newtheorem*{exs}{Examples}

\newtheorem*{subenonce}{}

\theoremstyle{yannthm2}
\newtheorem{exo}[defi]{Exercise}
\newtheorem*{exo*}{Exercise}
\newtheorem*{exos}{Exercises}
\newtheorem{rem}[defi]{Remark}
\newtheorem*{rem*}{Remark}
\newtheorem*{rems}{Remarks}

\newtheorem*{subenonce2}{}

}

\newenvironment{enonce}[1]{\begin{subenonce}[#1]}{\end{subenonce}}
\newenvironment{enonce2}[1]{\begin{subenonce2}[#1]}{\end{subenonce2}}

\newcommand{\bin}[2]{\tbinom{#1}{#2}}

\newcommand{\transp}[1]{#1^{\!\top}\!}



\title{The Third Bellman Equation}
\author{}

\newcommand{\drho}{\!\setminus\hspace{-.75em}\rho}
\newcommand{\ztar}{z^\mathrm{tar}}
\newcommand{\Nnewt}{N_\mathrm{steps/Newt}}
\newcommand{\noise}{\epsilon}

\begin{document}
\maketitle

\paragraph{Approximating $\Delta^{-1}$ via the Bellman--Newton equation.}
We learn an approximation $Z$ of $\Delta^{-1}$ via the Newton iteration
(V.Pan and J.Reif, \emph{Fast and efficient parallel solution of dense
linear systems}, 1989)
\begin{equation}
Z\gets 2Z-Z\Delta Z=2Z-Z^2+\gamma ZPZ
\end{equation}
which, if properly initialized, converges to $\Delta^{-1}$. A criterion
for correct initialization is if $\norm{\Id-Z\Delta}<1$ where $\norm{}$
is the operator norm. With $\Delta=\Id-\gamma P$ this is satisfied, for
instance, with $Z=\Id$ (for $\gamma<1$).

A stochastic version can be
\begin{equation}
Z\gets (1+\eta) Z-Z\tilde \Delta Z
\end{equation}
with a stochastic estimate $\tilde \Delta$ of $\Delta$ (obtained from a
transition observed in the MDP). Usually in RL we can only sample
transitions $s\to s'$ from some initial state $s$ distributed wrt some
distribution $\rho$ (which we may not control). We then parameterize
\begin{equation}
Z_{s_1s_2}\eqd \,\E_{s_2\sim \rho} z(s_1,s_2)
\end{equation}
namely, $z$ is the density of $Z$ with respect to $\rho(s_2)$.
Equivalently, in the matrix case, $Z=z\rho$ with $\rho$ the diagonal
matrix with entries $\rho$. Then the Newton update becomes
\begin{equation}
z\rho\gets 2z\rho-z\rho\Delta z\rho
\end{equation}
which is equivalent to
\begin{equation}
z\gets 2z-z\rho\Delta z
\end{equation}
and now $\rho \Delta$ is samplable:
\begin{equation}
\rho\Delta=\E_{s\sim \rho,\,s'\sim P(s'|s)}
\1_s\transp{(\1_s-\gamma\1_{s'})}
\end{equation}
so we can apply the stochastic version of the Newton update. Explicitly
this is
\begin{equation}
z(s_1,s_2)\gets (1+\eta) z(s_1,s_2)-
\E_{s\sim \rho,\,s'\sim P(s'|s)}z(s_1,s)\left(z(s,s_2)-\gamma
z(s',s_2)\right)
\end{equation}

In the tabular case, we can directly implement the above for all values
of $s_1,s_2$ each time we observe a transition $s\to s'$. (We can also
initialize to $\Id$.)

In the non-tabular case, we parameterize $z(s_1,s_2)$ by a machine
learning model. Then, given a current value $z_0$ of $z$, we to TD-like
steps towards the Newton update. More precisely, we define
\begin{equation}
\ztar\deq 2z_0-z_0\rho\Delta z_0
\end{equation}
and perform any number of SGD steps on the parameters of $z$ to reduce
the error $\E_{s_1\sim \rho,\,s_2\sim \rho}
\,(z(s_1,s_2)-\ztar(s_1,s_2))^2$, namely, we draw two states $s_1$ and
$s_2$ according to $\rho$, a transition $s\to s'$ with $s\sim \rho$, and
updated the parameter $\theta$ of $z$
by
\begin{equation}
\theta\gets \theta+\eta \partial_\theta z(s_1,s_2)\left(
2z_0(s_1,s_2)-z_0(s_1,s)\left(z_0(s,s_2)-\gamma z_0(s',s_2)\right)-z(s_1,s_2)
\right)
\end{equation}
which is an unbiased estimate of the gradient of the error.

We can sample several values of $s_1$ and $s_2$ for each sampled
transition $s\to s'$.  The distribution used on $s_1$ and $s_2$ need not
be the same distribution used for $s\to s'$ (though in the
parameterization by $Z-\Id$ this will be needed).

We perform $\Nnewt$ gradient steps before resetting the Bellman--Newton
objective using the current value of $z$ (namely, $z_0\gets z$ every
$\Nnewt$ steps). If $\Nnewt\gg 1$ and the machine learning model
converges, and $Z_0$ is initialized to $\Id$ (namely $z_0=1/\rho(s_2)$ at
startup) then this is guaranteed to converge.

\paragraph{What to do with $Z$.} Since $V=\Delta^{-1}R$, approximating
$\Delta^{-1}$ provides the ingredients for policy evaluation and policy
gradient. With the representation of $\Delta^{-1}$ as
$Z_{s_1s_2}=\E_{s_2\sim \rho} z(s_1,s_2)$, one has
\begin{equation}
V(s_1)=\E_{s_2\sim \rho}\, z(s_1,s_2) R(s_2)
\end{equation}
The representation of $V$ as an expectation may not be convenient. So if
$V$ is explicitly needed, we may need an auxiliary neural network to
learn $V(s_1)$ via the above (supervised learning). However, for policy
gradient the auxiliary network is not needed. Indeed, by the policy
gradient formula, on-policy (with $\mu$ the stationary distribution) we find
\begin{align}
\partial \E R&=
\E_{s\sim \mu,s'\sim P(s'|s)} \partial \ln
\pi(a|s) \left(
r(s,a,s')+\gamma V(s')-V(s)
\right)
\\&=
\E_{s\sim \mu,s'\sim P(s'|s),\,s_2\sim \rho} \partial \ln
\pi(a|s) \left(
r(s,a,s')+R(s_2)\left(\gamma z(s',s_2)-z(s,s_2)\right)
\right)
\end{align}
To exploit this, two possibilities. Either work online on $s\sim \mu$,
and use many samples from $s_2$ with a model for $R(s_2)$. Or (better),
work online on $s_2$ with an on-policy replay buffer on $s$, and update
several values of $s$ each time a reward is observed at $s_2$.

\paragraph{All rewards at once.} The relationship between $R$ and $V$ is
linear (for a fied policy). The generic representation of $Z$ above
requires computing a model of $V$ for each reward we want to work with.
We can directly exploit the linearity by parameterizing $z$ as
\begin{equation}
z(s_1,s_2)=\langle A(s_1),B(s_2)\rangle
\end{equation}
with some parameterized functions $A$ and $B$ with values in some vector
space with dot product $\langle\cdot,\cdot\rangle$. Intuitively, $A(s_1)$
represents the future of $s_1$, and $B(s_2)$ the past of $s_2$.
Then we have
\begin{equation}
V(s_1)=\langle A(s_1),\E_{s_2\sim \rho} B(s_2)R(s_2)\rangle
\end{equation}
so we can abuse notation and compute
\begin{equation}
B(R)\deq \E_{s_2\sim \rho} B(s_2)R(s_2)
\end{equation}
as the representation of $R$. Then
\begin{equation}
V(s_1)=\langle A(s_1),B(R)\rangle
\end{equation}

I'm not too sure about this representation. The rank of $Z$ is limited to
the dimension of the vector space, and this may cause problems. An
infinite-dimensional representation can be used, by having $A$ and $B$
take values in a space of $L^2$ functions over some probability space.
Namely, the dot product is
\begin{equation}
z(s_1,s_2)=\E_{\noise} A(s_1,\eps)B(s_2,\eps)
\end{equation}
and we can use as many samples of $\noise$ as we want. Obviously this
adds noise.

This computes the value functions of all rewards for a given policy.
Obviously when changing the policy, we may not change the policy in
several ways at once. For this we have to introduce targets (options) in
the policy.

\paragraph{All targets at once.} Let $\omega$ be an input to the policy,
which represents its current ``target''. We want to learn a policy with
input $\omega$, together with a function $\Omega$ that maps each state
$s$ to a target $\Omega(s)$, such that if we follow the policy with
target $\Omega(s)$, then we move towards $s$.

Concretely, we now have a function $z(s_1,s_2;\omega)$ which corresponds
to the inverse Laplacian of the MDP using policy $\omega$. It is learned
as above. Then we learn the parameters of $\pi(a|s,\omega)$ and of
$\Omega$ so as to maximize the time spent in $s$ when using policy
$\pi(\cdot|\Omega(s))$. Concretely, policy gradient is
\begin{equation}
\E_{s_1\sim \mu,\,a\sim \pi(a|s_1,\Omega(s)),\,s'\sim P(s'|s_1,a)}
\partial \ln \pi(a|s_1,\Omega(s))(\1_{s_1=s}+\gamma
z(s',s;\Omega(s))-z(s_1,s;\Omega(s)))
\end{equation}
(TODO: the $\1_{s_1=s}$ becomes a Dirac in the continuous case??)

I think the policy gradient above covers both the parameters of the
policy and the parameters of $\Omega$ (we can pretend the policy takes
$s$ as its target and that computing $\Omega$ is part of the policy).
But this conflicts with my previous intuition that we should train
$\Omega(s)$ to maximize $z(s',s;\Omega(s))$, which is the locally optimal
choice. I think we are missing some terms coming from the gradients of
$z(s_1,s;\Omega(s))$ when updating $\Omega$. TODO!! (Might be correct in
the end. Write that we want to maximize expected time spent in $s$ for
policy $\Omega(s)$. Write policy gradient for the policy with target $\Omega(s)$,
seeing $\Omega$ as part of the policy. Its policy gradient involves the
value function for policy $\Omega(s)$, which is given by $z(;\Omega(s))$,
but it does not involves the gradients of the value function.)

Problem here: $\mu$ has to be the invariant distribution of the
$\Omega(s)$-policy, so a bit complicated to apply online, we need to keep
the policy fixed until convergence... this implies having separate runs
with different targets. Or find an off-policy version... 

$\omega$ represents an option, but there is no risk of option collapse
because each option has a different target.

This still does not cover policies with arbitrary rewards $R$. We have to
find a representation $\Omega(R)$. We could try to enforce linearity of
$\Omega$, namely, train things such that the policy with parameter
$\omega=\E_{s\sim \rho} \Omega(s)R(s)$ maximizes $R$, for every $R$ in some
class. If the class includes Dirac, then 
the policy with parameter
$\Omega(s)$ indeed maximizes time spend in $s$. (This might help with the
Dirac in the policy gradient above.)

\end{document}
