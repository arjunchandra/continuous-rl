\documentclass[11pt,a4paper]{article}
%\usepackage[colorlinks=true,linkcolor=blue]{hyperref} --clashes with beamer...
\usepackage{cmap}
\usepackage[utf8x]{inputenc}
\usepackage[T1]{fontenc}
\usepackage[english]{babel}
\usepackage{amssymb,amsmath}
\usepackage{amsthm}
\usepackage{bbm}
\usepackage{microtype}
\usepackage{lmodern}

%To get bold small caps with lmodern
{
\rmfamily
\DeclareFontShape{T1}{lmr}{m}{sc}{<->ssub*cmr/m/sc}{}
\DeclareFontShape{T1}{lmr}{b}{sc}{<->ssub*cmr/b/sc}{}
\DeclareFontShape{T1}{lmr}{bx}{sc}{<->ssub*cmr/bx/sc}{}
}

% \newif\ifpdf
% \ifx\pdfoutput\undefined
%   \pdffalse
% \else
%   \pdfoutput=1
%   \pdftrue
% \fi
% 
% \ifpdf
%   \usepackage{aeguill}
%   \pdfcompresslevel9
% \else
% \fi
% 

%\usepackage{lmodern}

\newcommand{\augmentehauteur}[1]{
\addtolength{\topmargin}{-#1}
\addtolength{\textheight}{#1}
\addtolength{\textheight}{#1}
}

\newcommand{\augmentelargeur}[1]{
\addtolength{\evensidemargin}{-#1}
\addtolength{\oddsidemargin}{-#1}
\addtolength{\textwidth}{#1}
\addtolength{\textwidth}{#1}
}

\newcommand{\thmheadercommand}[1]{\textbf{\scshape{}#1.\\*}}

\newtheoremstyle{yannthm}{\topsep}{\topsep}{\slshape}{}{\scshape\bfseries}{.}{.5em}{%
\thmname{#1}\thmnumber{ #2}\thmnote{#3}%
}
\newtheoremstyle{yannthm2}{\topsep}{\topsep}{}{}{\scshape\bfseries}{.}{.5em}{%
\thmname{#1}\thmnumber{ #2}\thmnote{#3}%
}


\newcommand{\mynumero}{n°}


%\def\d{{\mathrm{d}}}
%\def\d{\operatorname{d}\!}
\def\d{\operatorname{d}\!{}}
%\def\d{\operatorname{}\!\mathrm{d}}

\def\N{{\mathbb{N}}}
\def\Z{{\mathbb{Z}}}
\def\Nstar{{\mathbb{N}^\star}}
\def\Q{{\mathbb{Q}}}
\def\R{{\mathbb{R}}}
\def\C{{\mathbb{C}}}

\renewcommand{\geq}{\geqslant}
\renewcommand{\leq}{\leqslant}

\renewcommand{\emptyset}{\varnothing}

\newcommand{\deq}{\mathrel{\mathop{:}}=}
\newcommand{\eqd}{=\mathrel{\mathop{:}}}

\newcommand{\from}{\colon} % correct ':' in f\from X \to Y
\newcommand{\st}{\mid} % set builder
%\newcommand{\st}{\mathrel{}\middle|\mathrel{}} % set builder

\def\eps{\varepsilon}
\renewcommand{\epsilon}{\varepsilon}
\renewcommand{\phi}{\varphi}

\def\ds{\displaystyle}

\DeclareMathOperator{\dist}{dist}
\DeclareMathOperator{\diam}{diam}
\DeclareMathOperator{\vol}{vol}
\DeclareMathOperator{\Ric}{Ric}

\DeclareMathOperator{\lap}{\Delta\!}
\DeclareMathOperator{\nab}{\nabla\!\!}
\DeclareMathOperator{\Hess}{Hess}

\DeclareMathOperator{\Ent}{Ent}
\DeclareMathOperator{\Var}{Var}
\DeclareMathOperator{\Cov}{Cov}
\let\oldPr\Pr
\renewcommand{\Pr}{\oldPr\nolimits}
\newcommand{\E}{\mathbb{E}}
\newcommand{\KL}[2]{\mathrm{KL}\!\left(#1 \,|\hspace{-.15ex}|\,#2\right)}

\DeclareMathOperator{\mult}{mult}
\DeclareMathOperator{\Card}{Card}
\DeclareMathOperator{\Aut}{Aut}
\DeclareMathOperator{\Epi}{Epi}
\DeclareMathOperator{\Spec}{Sp}
\DeclareMathOperator{\Ker}{Ker}
\DeclareMathOperator{\Img}{Im}
\DeclareMathOperator{\Tr}{Tr}
\DeclareMathOperator{\tr}{tr}
\DeclareMathOperator{\Tor}{Tor}
\DeclareMathOperator{\Ext}{Ext}
\DeclareMathOperator{\Hom}{Hom}
\DeclareMathOperator{\End}{End}
\DeclareMathOperator{\coker}{coker}
\DeclareMathOperator{\Id}{Id}
\DeclareMathOperator{\id}{id}
\DeclareMathOperator{\diag}{diag}


\newcommand{\abs}[1]{\left\lvert#1\right\rvert}
\newcommand{\norm}[1]{\left\lVert#1\right\rVert}
\newcommand{\scal}[2]{\left< \, #1 \mid #2 \, \right>}
\newcommand{\1}{\mathbbm{1}}

\newcommand{\ilim}[1]{\underset{#1}{\underrightarrow{\lim\vspace{.5ex}}}\,}
\newcommand{\plim}[1]{\underset{#1}{\underleftarrow{\lim\vspace{.5ex}}}\,}

\DeclareMathOperator*{\vlimsup}{\varlimsup}
\DeclareMathOperator*{\vliminf}{\varliminf}

\newcommand{\presgroup}[2]{\left\langle\,#1 \mid  #2\,\right\rangle}

\newcommand{\twopi}{2\hspace{-.23em}\pi}

\DeclareMathOperator*{\argmax}{arg\,max}
\DeclareMathOperator*{\argmin}{arg\,min}



%\newenvironment{dem}[1][]{\smallskip\noindent{\thmheadercommand{Proof#1}} \ignorespaces}{\hfill$\square$\medskip}

\newenvironment{dem}[1][]{\begin{proof}[\thmheadercommand{Proof#1}]~\newline\ignorespaces}{\end{proof}}

{
\theoremstyle{yannthm}
\newtheorem{defi}{Definition}
\newtheorem*{defi*}{Definition}
\newtheorem{prop}[defi]{Proposition}
\newtheorem*{prop*}{Proposition}
\newtheorem{thm}[defi]{Theorem}
\newtheorem*{thm*}{Theorem}
\newtheorem{lem}[defi]{Lemma}
\newtheorem*{lem*}{Lemma}
\newtheorem{cor}[defi]{Corollary}
\newtheorem*{cor*}{Corollary}
\newtheorem{ex}[defi]{Example}
\newtheorem*{ex*}{Example}
\newtheorem*{exs}{Examples}

\newtheorem*{subenonce}{}

\theoremstyle{yannthm2}
\newtheorem{exo}[defi]{Exercise}
\newtheorem*{exo*}{Exercise}
\newtheorem*{exos}{Exercises}
\newtheorem{rem}[defi]{Remark}
\newtheorem*{rem*}{Remark}
\newtheorem*{rems}{Remarks}

\newtheorem*{subenonce2}{}

}

\newenvironment{enonce}[1]{\begin{subenonce}[#1]}{\end{subenonce}}
\newenvironment{enonce2}[1]{\begin{subenonce2}[#1]}{\end{subenonce2}}

\newcommand{\bin}[2]{\tbinom{#1}{#2}}

\newcommand{\transp}[1]{#1^{\!\top}\!}





\begin{document}

We consider the Bellman equation for a policy $\pi$ in a finite MDP with
expected reward $R$ and transition probability matrix $P$ ($P$ includes $\pi$ and
the environment response). The Bellman equation is
\begin{equation}
(\Id-P)V=R
\end{equation}

We assume we can sample transitions $(s,a,r,s')$ according to some
distribution $\rho$ on $s$. In the on-policy case, $\rho$ will be the
stationary distribution. Typically $a$ has to be taken from $\pi$ (or a
distribution close to $\pi$ after resampling/by using only those $a$
sampled from $\pi$; this covers $\eps$-greedy) so that we do get
transitions from
$P$. (Alternatively one could work with pairs $(s,a)$ TODO.) Thus, we
have access to unbiased samples of $\rho P$ and $\rho R$:
\begin{equation}
\rho P=\E[\1_{s}\transp{\1_{s'}}\,],\qquad \rho R=\E[r(s,a,s')\1_{s}]
\end{equation}

If we parameterize $V$ by some parameter $\theta$, we can apply
parametric TD. The expected TD step is
\begin{equation}
\theta\gets \theta -\eta \,\transp{\Phi}\rho (V-PV-R)
\end{equation}
where 
\begin{equation}
\Phi\deq \frac{\partial V}{\partial \theta}
\end{equation}
is the Jacobian matrix of $V$ with respect to its parameter. The linear
case $V=\Phi \theta$ corresponds to constant $\Phi$.

A fixed point of TD solves
\begin{equation}
\transp{\Phi}\rho(\Id-P)V=\transp{\Phi}\rho R
\end{equation}
and if the system is overparameterized ($\Phi$ invertible) and
well-sampled ($\rho>0$) this implies $(\Id-P)V=R$. All terms in this
equation are \emph{samplable}: we can sample from $\rho R$ and from
$\rho(\Id-P)$.

A local minimizer of the $L^2(\rho)$ error between $V$ and
$(\Id-P)^{-1}R$ solves
\begin{equation}
\transp{\Phi} \rho V=\transp{\Phi}\rho (\Id-P)^{-1} R
\end{equation}
which is not quite the same as the above. A local minimizer of the
$L^2(\rho)$ error between $(\Id-P)V$ and $R$ is
\begin{equation}
\transp{\Phi}\transp{(\Id-P)}\rho(\Id-P)V=\transp{\Phi}\transp{(\Id-P)}\rho R
\end{equation}
and in the linear case $V=\Phi\theta$, this is the standard solution to
the linear regression problem $(\Id-P)\Phi\theta\approx R$ in
$L^2(\rho)$.

The Dirichlet norm associated with $\rho$ is $\E_{ss'}
(f(s)-f(s'))^2$. Its matrix is $\rho+\rho'-\rho P-\transp{(\rho P)}$ where
$\rho'$ is the distribution of $s'$.
A local minimizer of the Dirichlet error between $V$ and
$(\Id-P)^{-1}R$ solves
\begin{equation}
\transp{\Phi} (\rho+\rho'-\rho P-\transp{(\rho P)})(V-(\Id-P)^{-1}R)=0
\end{equation}
If $\rho$ is the stationary distribution $\mu$ and $P$ is reversible, one
has $\transp{(\mu P)}=\mu P$ so $\rho+\rho'-\rho P-\transp{(\rho
P)}=2\mu(\Id-P)$ and this is equivalent to
\begin{equation}
\transp{\Phi}\mu ((\Id-P)V-R)=0
\end{equation}
namely, the same as a fixed point of TD.

\bigskip

TD is the only one of those fixed point equations that is
\emph{samplable} given that we can sample $\rho$, $\rho P$ and $\rho R$.
\footnote{Actually the minimizer of the $L^2(\rho^2)$ error between $R$
and $(\Id-P)V$ is also samplable:
\begin{equation}
\transp{\Phi}\transp{(\Id-P)}\rho^2(\Id-P)V=\transp{\Phi}\transp{(\Id-P)}\rho^2
R
\end{equation}
however, the $\rho^2$ is a bit weird...
}
Thus, from now on we are going to focus on solving
\begin{equation}
\transp{\Phi}\rho (\Id-P)V=\transp{\Phi}\rho R
\end{equation}
using various SGD-like solvers.

We now switch to the linear case and will de-linearize later. This
becomes
\begin{equation}
\transp{\Phi}\rho (\Id-P)\Phi \theta=\transp{\Phi}\rho R
\end{equation}
to be solved in $\theta$.

\paragraph{Solving $A\theta=b$ where $A$ and $b$ are samplable.} An
obvious way to solve $A\theta=b$ via a stochastic gradient method is
\begin{equation}
\theta\gets \theta - \eta (A\theta-b)
\end{equation}
using samples for $A$ and $b$.
This is TD. If $A$ is symmetric positive definite, this is also the
gradient descent of the loss function
$\transp{(A\theta-b)}A^{-1}(A\theta-b)$. However, we want to apply this
to matrices $A$ that are not necessarily of this type.

This works only if $A$ is \emph{stable}.


\begin{defi}
A matrix $A$ is \emph{stable} if one of the following equivalent
conditions are satisfied:
\begin{itemize}
\item All the eigenvalues of $A$ have positive real part.
\item The matrix $(\Id-\eps A)$ has spectral radius $<1$ for small enough
$\eps$.
\item The differential equation $\theta'=-A\theta$ converges to $0$ for
any initial value.
\item There exists a symmetric, positive definite matrix $L$ (Lyapunov
function) such that $\transp{\theta}L\theta$ is decreasing along the
solutions of the differential equation $\theta'=-A\theta$.
\item There exists a symmetric, positive definite matrix $L$ such that
\begin{equation}
\transp{\theta} L A \theta>0
\end{equation}
for all $\theta\neq 0$. (This is the same $L$ as in the previous
condition.)
\end{itemize}
\end{defi}

In particular, a symmetric, definite positive matrix is stable. Since the
solution of $\theta_t'=-A\theta_t$ is $\theta_t=e^{-tA}\theta_0$, a
Lyapunov function that works is $\transp{\theta}L\theta=\int_{t \geq 0}
\norm{\theta_t}^2$, namely
$L=\int_{t\geq 0} \transp{(e^{-t A})}
e^{-tA}$.

\end{document}
