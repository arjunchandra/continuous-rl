%%%%%%%% ICML 2018 EXAMPLE LATEX SUBMISSION FILE %%%%%%%%%%%%%%%%%

\documentclass{article}

% Recommended, but optional, packages for figures and better typesetting:
\usepackage{amsmath, amssymb}
\usepackage{microtype}
\usepackage{graphicx}
\usepackage{subfigure}
\usepackage{booktabs} % for professional tables
\usepackage{xcolor}

% hyperref makes hyperlinks in the resulting PDF.
% If your build breaks (sometimes temporarily if a hyperlink spans a page)
% please comment out the following usepackage line and replace
% \usepackage{icml2018} with \usepackage[nohyperref]{icml2018} above.
\usepackage{hyperref}

%! TEX root = main.tex
\usepackage{xcolor}
\usepackage{algorithm}
\usepackage{algpseudocode}
\usepackage{hyperref}

%\newcommand{\deltat}{\delta\hspace*{-.3mm}t}
\newcommand{\deltat}{\ensuremath{\delta\hspace{-.06em}t}}
\newcommand{\bigO}[1]{O (#1)}
\newcommand{\reward}{\tilde{r}}
\newcommand{\actionspace}{\mathcal{A}}
\newcommand{\statespace}{\mathcal{S}}
\newcommand{\NDC}[1]{\color{red}NDC:#1\color{black}}

\newcommand{\pfield}{\mathcal{F}_t}
\newcommand{\Ep}[1]{\BE\left[#1\mid\pfield\right]}
\newcommand{\diag}[1]{\mathrm{\textbf{diag}}(#1)}
\newcommand{\lmax}{\lambda_{\mathrm{max}}}



\newcommand{\mynumero}{n°}


%\def\d{{\mathrm{d}}}
%\def\d{\operatorname{d}\!}
\def\d{\operatorname{d}\!{}}
%\def\d{\operatorname{}\!\mathrm{d}}

\def\N{{\mathbb{N}}}
\def\Z{{\mathbb{Z}}}
\def\Nstar{{\mathbb{N}^\star}}
\def\Q{{\mathbb{Q}}}
\def\R{{\mathbb{R}}}
\def\C{{\mathbb{C}}}

\renewcommand{\geq}{\geqslant}
\renewcommand{\leq}{\leqslant}

\renewcommand{\emptyset}{\varnothing}

\newcommand{\deq}{\mathrel{\mathop{:}}=}
\newcommand{\eqd}{=\mathrel{\mathop{:}}}

\newcommand{\from}{\colon} % correct ':' in f\from X \to Y
\newcommand{\st}{\mid} % set builder
%\newcommand{\st}{\mathrel{}\middle|\mathrel{}} % set builder

\def\eps{\varepsilon}
\renewcommand{\epsilon}{\varepsilon}
\renewcommand{\phi}{\varphi}

\def\ds{\displaystyle}

\DeclareMathOperator{\dist}{dist}
\DeclareMathOperator{\diam}{diam}
\DeclareMathOperator{\vol}{vol}
\DeclareMathOperator{\Ric}{Ric}

\DeclareMathOperator{\lap}{\Delta\!}
\DeclareMathOperator{\nab}{\nabla\!\!}
\DeclareMathOperator{\Hess}{Hess}

\DeclareMathOperator{\Ent}{Ent}
\DeclareMathOperator{\Var}{Var}
\DeclareMathOperator{\Cov}{Cov}
\let\oldPr\Pr
\renewcommand{\Pr}{\oldPr\nolimits}
\newcommand{\E}{\mathbb{E}}
\newcommand{\KL}[2]{\mathrm{KL}\!\left(#1 \,|\hspace{-.15ex}|\,#2\right)}

\DeclareMathOperator{\mult}{mult}
\DeclareMathOperator{\Card}{Card}
\DeclareMathOperator{\Aut}{Aut}
\DeclareMathOperator{\Epi}{Epi}
\DeclareMathOperator{\Spec}{Sp}
%\DeclareMathOperator{\Ker}{Ker}
\DeclareMathOperator{\Img}{Im}
\DeclareMathOperator{\Tr}{Tr}
\DeclareMathOperator{\tr}{tr}
\DeclareMathOperator{\Tor}{Tor}
\DeclareMathOperator{\Ext}{Ext}
\DeclareMathOperator{\Hom}{Hom}
\DeclareMathOperator{\End}{End}
\DeclareMathOperator{\coker}{coker}
\DeclareMathOperator{\Id}{Id}
\DeclareMathOperator{\id}{id}
%\DeclareMathOperator{\diag}{diag}


\newcommand{\abs}[1]{\left\lvert#1\right\rvert}
\newcommand{\norm}[1]{\left\lVert#1\right\rVert}
\newcommand{\scal}[2]{\left< \, #1 \mid #2 \, \right>}
\newcommand{\1}{\mathbbm{1}}

\newcommand{\ilim}[1]{\underset{#1}{\underrightarrow{\lim\vspace{.5ex}}}\,}
\newcommand{\plim}[1]{\underset{#1}{\underleftarrow{\lim\vspace{.5ex}}}\,}

\DeclareMathOperator*{\vlimsup}{\varlimsup}
\DeclareMathOperator*{\vliminf}{\varliminf}

\newcommand{\presgroup}[2]{\left\langle\,#1 \mid  #2\,\right\rangle}

\newcommand{\twopi}{2\hspace{-.23em}\pi}

%\DeclareMathOperator*{\argmax}{arg\,max}
%\DeclareMathOperator*{\argmin}{arg\,min}



% Use the following line for the initial blind version submitted for review:
\usepackage{icml2018}

% If accepted, instead use the following line for the camera-ready submission:
%\usepackage[accepted]{icml2018}

% The \icmltitle you define below is probably too long as a header.
% Therefore, a short form for the running title is supplied here:
\icmltitlerunning{Deep Relativistic Advantage Updating}

\begin{document}

\twocolumn[
\icmltitle{There are no small Advantages:
	Reinforcement learning in near continuous time
	with Deep RelAtive Mixture of Advantages
}


% It is OKAY to include author information, even for blind
% submissions: the style file will automatically remove it for you
% unless you've provided the [accepted] option to the icml2018
% package.

% List of affiliations: The first argument should be a (short)
% identifier you will use later to specify author affiliations
% Academic affiliations should list Department, University, City, Region, Country
% Industry affiliations should list Company, City, Region, Country

% You can specify symbols, otherwise they are numbered in order.
% Ideally, you should not use this facility. Affiliations will be numbered
% in order of appearance and this is the preferred way.
\icmlsetsymbol{equal}{*}

\begin{icmlauthorlist}
%\icmlauthor{Aeiau Zzzz}{equal,to}
%\icmlauthor{Bauiu C.~Yyyy}{equal,to,goo}
%\icmlauthor{Cieua Vvvvv}{goo}
%\icmlauthor{Iaesut Saoeu}{ed}
%\icmlauthor{Fiuea Rrrr}{to}
%\icmlauthor{Tateu H.~Yasehe}{ed,to,goo}
%\icmlauthor{Aaoeu Iasoh}{goo}
%\icmlauthor{Buiui Eueu}{ed}
%\icmlauthor{Aeuia Zzzz}{ed}
%\icmlauthor{Bieea C.~Yyyy}{to,goo}
%\icmlauthor{Teoau Xxxx}{ed}
%\icmlauthor{Eee Pppp}{ed}
\end{icmlauthorlist}

%\icmlaffiliation{to}{Department of Computation, University of Torontoland, Torontoland, Canada}
%\icmlaffiliation{goo}{Googol ShallowMind, New London, Michigan, USA}
%\icmlaffiliation{ed}{School of Computation, University of Edenborrow, Edenborrow, United Kingdom}
%
%\icmlcorrespondingauthor{Cieua Vvvvv}{c.vvvvv@googol.com}
%\icmlcorrespondingauthor{Eee Pppp}{ep@eden.co.uk}

% You may provide any keywords that you
% find helpful for describing your paper; these are used to populate
% the "keywords" metadata in the PDF but will not be shown in the document
\icmlkeywords{Reinforcement learning}

\vskip 0.3in
]

% this must go after the closing bracket ] following \twocolumn[

% This command actually creates the footnote in the first column
% listing the affiliations and the copyright notice.
% The command takes one argument, which is text to display at the start of the footnote.
% The \icmlEqualContribution command is standard text for equal contribution.
% Remove it (just {}) if you do not need this facility.

%\printAffiliationsAndNotice{}  % leave blank if no need to mention equal contribution
\printAffiliationsAndNotice{\icmlEqualContribution} % otherwise use the standard text.

\begin{abstract}

\end{abstract}

\section{Introduction}
\label{sec:intro}

\section{Related work}
\label{sec:related}

\section{Continuous reinforcement learning and scaling problems}
\label{sec:continous}
In the following section, a restrictive framework for near continuous
time reinforcement learning is introduced, and the main challenges of this
setup are explained. \RKC{This is not what we want to write: we want to explain
why current reinforcement learning practices are ill suited for this framework}

\subsection{Framework}
Consider a fully observable \emph{Markov Decision Process} defined by the following
transitions and rewards
\begin{align}
	s_{t + \deltat} &= s_t + F(s_t, a_t) \deltat + \Sigma(s_t, a_t) \deltabt\\
	r_{t} &= \reward(s_t, a_t) \deltat + \Sigma_r(s_t, a_t) \deltabt^r
\end{align}
where the $s$'s are $n$-dimensional vectors, $F(s_t, a_t)$ is a $n$-dimensional
drift coeficient, $\Sigma(s_t, a_t)$ is a $n \times n$ diffusion matrix,
$\deltat$ is a small but finite discretization step, and $\deltabt$ is a sample
from $\gauss(0, \deltat I)$.

This framework is quite general, and most discretized deterministic continuous
control environments fall in this framework, with $\Sigma = 0$ and $\Sigma_r =
0$, as well as many stochastic control environments.

A deterministic policy $\pi$ is defined as a deterministic mapping from state
space to action space\footnote{We restrict ourselves to deterministic policies.
	In continuous time, defining stochastic policies with well defined
	state-value function is non trivial. In the discrete case, the
distribution of actions of a policy given a state only depends on the previous
state, and not on previous actions. In continuous time, such policies are akin
to white noise, and do not define stochastic processes.}.

Under a deterministic policy $\pi$, one can define the \emph{state value function} $V^\pi(s)$
and \emph{state action value function} $Q^\pi(s, a)$ as
\begin{align}

\end{align}

\section{Multi-chain continuous environments and biscale advantage}
\label{sec:multi-chain}

\section{Deep RelAtivistic Mixture of Advantage learning}
\label{sec:algo}

\section{Experiments}
\label{sec:exp}

\section{Conclusion}
\label{sec:concl}

\bibliography{icml_drau.bib}
\bibliographystyle{icml2018}
\end{document}


% This document was modified from the file originally made available by
% Pat Langley and Andrea Danyluk for ICML-2K. This version was created
% by Iain Murray in 2018. It was modified from a version from Dan Roy in
% 2017, which was based on a version from Lise Getoor and Tobias
% Scheffer, which was slightly modified from the 2010 version by
% Thorsten Joachims & Johannes Fuernkranz, slightly modified from the
% 2009 version by Kiri Wagstaff and Sam Roweis's 2008 version, which is
% slightly modified from Prasad Tadepalli's 2007 version which is a
% lightly changed version of the previous year's version by Andrew
% Moore, which was in turn edited from those of Kristian Kersting and
% Codrina Lauth. Alex Smola contributed to the algorithmic style files.
