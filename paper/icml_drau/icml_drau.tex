%%%%%%%% ICML 2018 EXAMPLE LATEX SUBMISSION FILE %%%%%%%%%%%%%%%%%

\documentclass{article}

% Recommended, but optional, packages for figures and better typesetting:
\usepackage{amsmath, amssymb}
\usepackage{mathtools}
\usepackage{microtype}
\usepackage{graphicx}
\usepackage{subfigure}
\usepackage{booktabs} % for professional tables
\usepackage{xcolor}
\usepackage{algorithmic}

% hyperref makes hyperlinks in the resulting PDF.
% If your build breaks (sometimes temporarily if a hyperlink spans a page)
% please comment out the following usepackage line and replace
% \usepackage{icml2018} with \usepackage[nohyperref]{icml2018} above.
\usepackage{hyperref}

%! TEX root = main.tex
\usepackage{xcolor}
\usepackage{algorithm}
\usepackage{algpseudocode}
\usepackage{hyperref}

%\newcommand{\deltat}{\delta\hspace*{-.3mm}t}
\newcommand{\deltat}{\ensuremath{\delta\hspace{-.06em}t}}
\newcommand{\bigO}[1]{O (#1)}
\newcommand{\reward}{\tilde{r}}
\newcommand{\actionspace}{\mathcal{A}}
\newcommand{\statespace}{\mathcal{S}}
\newcommand{\NDC}[1]{\color{red}NDC:#1\color{black}}

\newcommand{\pfield}{\mathcal{F}_t}
\newcommand{\Ep}[1]{\BE\left[#1\mid\pfield\right]}
\newcommand{\diag}[1]{\mathrm{\textbf{diag}}(#1)}
\newcommand{\lmax}{\lambda_{\mathrm{max}}}



\newcommand{\mynumero}{n°}


%\def\d{{\mathrm{d}}}
%\def\d{\operatorname{d}\!}
\def\d{\operatorname{d}\!{}}
%\def\d{\operatorname{}\!\mathrm{d}}

\def\N{{\mathbb{N}}}
\def\Z{{\mathbb{Z}}}
\def\Nstar{{\mathbb{N}^\star}}
\def\Q{{\mathbb{Q}}}
\def\R{{\mathbb{R}}}
\def\C{{\mathbb{C}}}

\renewcommand{\geq}{\geqslant}
\renewcommand{\leq}{\leqslant}

\renewcommand{\emptyset}{\varnothing}

\newcommand{\deq}{\mathrel{\mathop{:}}=}
\newcommand{\eqd}{=\mathrel{\mathop{:}}}

\newcommand{\from}{\colon} % correct ':' in f\from X \to Y
\newcommand{\st}{\mid} % set builder
%\newcommand{\st}{\mathrel{}\middle|\mathrel{}} % set builder

\def\eps{\varepsilon}
\renewcommand{\epsilon}{\varepsilon}
\renewcommand{\phi}{\varphi}

\def\ds{\displaystyle}

\DeclareMathOperator{\dist}{dist}
\DeclareMathOperator{\diam}{diam}
\DeclareMathOperator{\vol}{vol}
\DeclareMathOperator{\Ric}{Ric}

\DeclareMathOperator{\lap}{\Delta\!}
\DeclareMathOperator{\nab}{\nabla\!\!}
\DeclareMathOperator{\Hess}{Hess}

\DeclareMathOperator{\Ent}{Ent}
\DeclareMathOperator{\Var}{Var}
\DeclareMathOperator{\Cov}{Cov}
\let\oldPr\Pr
\renewcommand{\Pr}{\oldPr\nolimits}
\newcommand{\E}{\mathbb{E}}
\newcommand{\KL}[2]{\mathrm{KL}\!\left(#1 \,|\hspace{-.15ex}|\,#2\right)}

\DeclareMathOperator{\mult}{mult}
\DeclareMathOperator{\Card}{Card}
\DeclareMathOperator{\Aut}{Aut}
\DeclareMathOperator{\Epi}{Epi}
\DeclareMathOperator{\Spec}{Sp}
%\DeclareMathOperator{\Ker}{Ker}
\DeclareMathOperator{\Img}{Im}
\DeclareMathOperator{\Tr}{Tr}
\DeclareMathOperator{\tr}{tr}
\DeclareMathOperator{\Tor}{Tor}
\DeclareMathOperator{\Ext}{Ext}
\DeclareMathOperator{\Hom}{Hom}
\DeclareMathOperator{\End}{End}
\DeclareMathOperator{\coker}{coker}
\DeclareMathOperator{\Id}{Id}
\DeclareMathOperator{\id}{id}
%\DeclareMathOperator{\diag}{diag}


\newcommand{\abs}[1]{\left\lvert#1\right\rvert}
\newcommand{\norm}[1]{\left\lVert#1\right\rVert}
\newcommand{\scal}[2]{\left< \, #1 \mid #2 \, \right>}
\newcommand{\1}{\mathbbm{1}}

\newcommand{\ilim}[1]{\underset{#1}{\underrightarrow{\lim\vspace{.5ex}}}\,}
\newcommand{\plim}[1]{\underset{#1}{\underleftarrow{\lim\vspace{.5ex}}}\,}

\DeclareMathOperator*{\vlimsup}{\varlimsup}
\DeclareMathOperator*{\vliminf}{\varliminf}

\newcommand{\presgroup}[2]{\left\langle\,#1 \mid  #2\,\right\rangle}

\newcommand{\twopi}{2\hspace{-.23em}\pi}

%\DeclareMathOperator*{\argmax}{arg\,max}
%\DeclareMathOperator*{\argmin}{arg\,min}



% Use the following line for the initial blind version submitted for review:
\usepackage{icml2018}

% If accepted, instead use the following line for the camera-ready submission:
%\usepackage[accepted]{icml2018}

% The \icmltitle you define below is probably too long as a header.
% Therefore, a short form for the running title is supplied here:
\icmltitlerunning{Deep Relativistic Advantage Updating}

\begin{document}

\twocolumn[
\icmltitle{There are no small Advantages:
	Reinforcement learning in near continuous time
	with Deep RelAtive Mixture of Advantages
}


% It is OKAY to include author information, even for blind
% submissions: the style file will automatically remove it for you
% unless you've provided the [accepted] option to the icml2018
% package.

% List of affiliations: The first argument should be a (short)
% identifier you will use later to specify author affiliations
% Academic affiliations should list Department, University, City, Region, Country
% Industry affiliations should list Company, City, Region, Country

% You can specify symbols, otherwise they are numbered in order.
% Ideally, you should not use this facility. Affiliations will be numbered
% in order of appearance and this is the preferred way.
\icmlsetsymbol{equal}{*}

\begin{icmlauthorlist}
%\icmlauthor{Aeiau Zzzz}{equal,to}
%\icmlauthor{Bauiu C.~Yyyy}{equal,to,goo}
%\icmlauthor{Cieua Vvvvv}{goo}
%\icmlauthor{Iaesut Saoeu}{ed}
%\icmlauthor{Fiuea Rrrr}{to}
%\icmlauthor{Tateu H.~Yasehe}{ed,to,goo}
%\icmlauthor{Aaoeu Iasoh}{goo}
%\icmlauthor{Buiui Eueu}{ed}
%\icmlauthor{Aeuia Zzzz}{ed}
%\icmlauthor{Bieea C.~Yyyy}{to,goo}
%\icmlauthor{Teoau Xxxx}{ed}
%\icmlauthor{Eee Pppp}{ed}
\end{icmlauthorlist}

%\icmlaffiliation{to}{Department of Computation, University of Torontoland, Torontoland, Canada}
%\icmlaffiliation{goo}{Googol ShallowMind, New London, Michigan, USA}
%\icmlaffiliation{ed}{School of Computation, University of Edenborrow, Edenborrow, United Kingdom}
%
%\icmlcorrespondingauthor{Cieua Vvvvv}{c.vvvvv@googol.com}
%\icmlcorrespondingauthor{Eee Pppp}{ep@eden.co.uk}

% You may provide any keywords that you
% find helpful for describing your paper; these are used to populate
% the "keywords" metadata in the PDF but will not be shown in the document
\icmlkeywords{Reinforcement learning}

\vskip 0.3in
]

% this must go after the closing bracket ] following \twocolumn[

% This command actually creates the footnote in the first column
% listing the affiliations and the copyright notice.
% The command takes one argument, which is text to display at the start of the footnote.
% The \icmlEqualContribution command is standard text for equal contribution.
% Remove it (just {}) if you do not need this facility.

%\printAffiliationsAndNotice{}  % leave blank if no need to mention equal contribution
%\printAffiliationsAndNotice{\icmlEqualContribution} % otherwise use the standard text.

\begin{abstract}
	Despite remarkable successes, \emph{Deep Reinforcement learning} remains sensitive to
	hyperparameterization, implementation details and small environment changes~(\citealt{drl_matter}, \citealt{drl_matter_bis}). 
	Overcoming this sensitivity is key to making DRL applicable to real world problems.
	In this paper, we study the sensitivity of Deep Reinforcement learning algorithms to
	time discretization. More precisely, we show that approaches based on estimations of the
	Q-function, e.g. \emph{Deep Q-learning}~\cite{dqn} and \emph{Deep deterministic policy gradient}~\cite{ddpg}
	are sensitive to variations of time discretization. We further show that a simple reparameterization of the
	Q-function as a sum of a state value term and a \emph{small} action dependent advantage term yields an algorithm
	much more resilient to variations of time discretization.
\end{abstract}

\section{Introduction}
\label{sec:intro}
In recent years, deep learning based approaches to reinforcement learning have
thrived and provided impressive results in a variety of domains, achieving superhuman
performances with no expert knowledge in perfect information zero-sum
games~\cite{alphazero}, reaching level of top players in video
games~(\citealt{openai_five}, \citealt{dqn}), or learning dexterous manipulation
from scratch without demonstrations~\cite{hand_control}.

In spite of those successes, \emph{Deep reinforcement learning} approaches are
sensitive to a number of factors, including hyperparameterization,
implementation details or small changes in the environment
parameters~(\citealt{drl_matter}, \citealt{drl_matter_bis}). This sensitivity,
along with sample inefficiency, prevents DRL from being applied in most real
world settings. High sensitivity to environment parameters notably prevents
transfering abilities from imperfect simulators to real world scenarios.

In this paper we focus our attention on the sensitivity to time discretization
of DRL approaches, i.e. what happens when an agent receives $50$ observations
and is expected to take $50$ actions per second instead of $10$. One expects
that decreasing time discretization, or equivalently shortening reaction time,
should only improve agent performances. This is not what happens in practice
for approaches based on estimation of state action value functions, e.g.
\emph{Deep Q-learning} (DQN~\citep{dqn} and \emph{Deep deterministic policy
gradient} (DDPG~\citep{ddpg}). This is shown experimentally in Sec.~\ref{sec:exp}.

We relate this sensitivity to the fact that, as the discretization timestep
decreases, the effect of individual actions on the total return decreases, and
can vanish when the discretization timestep becomes infinitesimal. The natural
framework to study this phenomenon is Continuous time reinforcement learning 
(Sec.~\ref{sec:continous}). 

To cope for this effect, we introduce a
parameterization of the Q-function as a sum of a state value term and a
\emph{small} action dependent advantage term (Sec.~\ref{subsec:reparam}). This
reparameterized Q approximation can be trained with deep variants of Q-learning
(Sec.~\ref{subsec:algorithm}).  To properly scale w.r.t. time discretization,
particular attention has to be taken regarding both the exploration method, and
the scaling of learning rates (Sec.~\ref{subsec:lr}, Sec.~\ref{subsec:explo}).
The resulting algorithm is shown to provide near perfect invariance to time
discretization on simple environments, and much better invariance properties
than vanilla DQN or DDPG on more complex environments (Sec.~\ref{sec:exp}).

\section{Related work}
\label{sec:related}
Continuous and near continuous reinforcement learning, which is at the core of
our approach has been thoroughly studied in the deterministic case
in~\cite{adv_upd} or \cite{cont_rl}, but has not been applied in the context of
deep reinforcement learning.

The approach presented in this paper is closely related to \cite{adv_upd} and
\cite{cont_rl}. \cite{adv_upd} already separated the learning procedure of the
value and advantage functions, and introduced proper scalings between the two
components. We extend on \cite{adv_upd} in three directions. We provide a
formal argument on why the advantage contribution of the Q-function is of order
$\bigO(\deltat)$, irrelevant of the stochasticity of the environment, in a quite
general framework. We show that, provided with well suited learning rates and
exploration method, Advantage updating provide near invariance to change of time
discretization. We show that Advantage updating is viable in the context of deep
reinforcement learning, while not using the true gradient of the residual error.

More recently, \cite{dueling_nets} also introduced a parameterization separating
the value and the advantage components of the Q-function. We additionally
introduce a natural scaling coeficient between the two, that strongly improves
the time discretization invariance properties.

\section{Near continuous time reinforcement learning}
\label{sec:continous}
% In this section, a continuous reinforcement learning framework is introduced,
% and it is shown that under this framework, the state-action value function does
% not depend on the action. We further show that in near continous domains, when
% the discretization timestep $\deltat$ is small, the difference in state-action
% value between two different actions in a fixed state is of order $\bigO(\deltat)$.
TODO. Main ideas: near continuous time environments (control, mujoco, video games, ...)
There is a time discretization $\deltat$ given by the problem. The uderlying process does not depend of $\deltat$. Need to see what is happening when changing $\deltat$. The problem is invariant to $\deltat$. The quantities we are looking at should be invariant to. The learning procedure should be invariant. Looking at the limits when $\deltat \rightarrow 0$. In simulated environments, changing $\deltat$ only means to change the FPS. In real environments, for example with robots, this could mean changing the quality of the sensors and motors.

\subsection{Framework}

Let assume that we have a Markov Decision Process (TODO:Cite?) $\langle {\cal S}, {\cal A}, T_{\deltat}, r, \gamma\rangle$. ${\cal S}$ is the state-space, which is supposed to be continuous (TODO:not the good formulation). The action space is TODO.

We assume that the transition is continuous. This means that ... TODO. Discussion on the hypothesis, in which environments it is true. Is the condition $\forall \deltat, \forall s, \E[\|s_{k+1} - s_k \||s_k = s] \leq ?$. 

In what follows, $\dt$ denotes a truly infinitesimal time interval while $\deltat$
denotes a small, but non infinitesimal time interval. Similarily, $\dbt$ denotes
an infinitesimal (multi-dimensional) brownian step, while $\deltabt$ denotes its
discretized equivalent. Informally, $\dbt$ is to be thought of as $\gauss(0, \sqrt{\dt}^2)$,
and $\deltabt$ as $\gauss(0, \sqrt{\deltat}^2)$, where $\gauss(\mu, \sigma)$ denotes a gaussian
random variable of mean $\mu$ and standard deviation (or covariance matrix in the multi-dimensional case)
$\sigma$.

Consider a fully observable \emph{Markov Decision Process} defined by the following
transitions and rewards
\begin{align}
	ds(s, a) &= F(s, a) dt + \Sigma(s, a) \dbt\\
	dr(s, a) &= \reward(s, a) dt + \Sigma_r(s, a) \dbt.
\end{align}
The equation on $s$ is that of a diffusion process, with drift coeficient $F$
and diffusion matrix $\Sigma$.  This framework is only moderately restrictive.
For example, deterministic continuous control environments fall in this
framework, with $\Sigma = 0$ and $\Sigma_r = 0$. Limitations of the framework
are discussed in Sec.~\ref{sec:limitations}.
This framework is naturally discretized to arbitrary $\deltat$ by replacing all
$d$'s with $\delta$'s.

A continuous policy $\pi$ is defined as a deterministic mapping from states to
actions.  In continuous time, defining stochastic policies with well defined
state-value functions requires incorporating information from previous action into
the state to obtain a temporally coherent policy. To avoid such inconvenience, the
framework is restricted to deterministic policies. This is a mild constraint, as this
only enforces determinism of the exploitation policy and not of exploration policies.
In what follows, $\tau\sim\pi$ denotes a trajectory of states and actions sampled
according to policy $\pi$, with $s_0$ sampled according to an arbitrary initial distribution
on states $\rho_0$, and $dr_{t} = dr(s_t, a_t)$.

Under a deterministic policy $\pi$, one can define the $\gamma$ discounted
\emph{state value function} $V^\pi(s)$ and \emph{state action value function}
$Q^\pi(s, a)$ as
\begin{align}
	V^\pi(s) &= \E_{\tau\sim\pi}\left[
		\int\limits_{t=0}^\infty \gamma^{t}
		dr_{t} \mid s_0 = s
	\right]\\
	Q^\pi(s, a) &= \E_{\tau\sim\pi}\left[
		\int\limits_{t=0}^\infty \gamma^{t}
		dr_{t} \mid s_0 = s, a_0 = a
	\right]
\end{align}
which are naturally discretized as
\begin{align}
	V^\pi(s) &= \E_{\tau\sim\pi}\left[
		\sum\limits_{k=0}^\infty \gamma^{k\deltat}
		\delta r_{k\deltat} \mid s_0 = s
	\right]\\
	Q^\pi(s, a) &= \E_{\tau\sim\pi}\left[
		\sum\limits_{k=0}^\infty \gamma^{k\deltat}
		\delta r_{k\deltat} \mid s_0 = s, a_0 = a
	\right].
\end{align}
Both $V^\pi$ and $Q^\pi$ verify Bellman equations, informally
\begin{align}
	V^\pi(s) &= \reward(s, \pi(s))\dt + \gamma^{\dt} \E_{ds \sim ds(s, \pi(s))} V^\pi(s + ds)\nonumber\\
	Q^\pi(s, a) &= \reward(s, a)\dt + \gamma^{\dt} \E_{ds(s, a)} V^\pi(s + ds(s, a))
	\label{eq:cont_bell}
\end{align}
which in continuous time translates, using Ito's lemma, into a Hamilton Jacobi equation
\begin{align}
	\reward + \partial_s V^\pi \cdot f + \frac{1}{2} \text{tr}\left(\Sigma^T\partial^2_{s^2} V^\pi\Sigma\right) = (1 - \gamma) V^\pi(s).
\end{align}

\subsection{Independence of $Q^\pi$ on $a$ in continuous time}
Consider a discretization of Eq.~\eqref{eq:cont_bell} with time interval
$\deltat \ll~1$, Ito's lemma yield
\begin{align}
	Q^\pi(s, a) &= \reward(s, a) \deltat + \gamma^{\deltat} \E_{\delta s \sim \delta s(s, a)}V^\pi(s + \delta s)\nonumber\\
		    &= \reward(s, a) \deltat + \gamma^{\deltat} V^\pi(s) +
		     \gamma^{\deltat} \partial_s V^\pi(s)f(s, a) \deltat\nonumber\\
		    &+ \frac{1}{2}\text{tr}\left(\Sigma^T\partial^2_{s^2}V^\pi(s)\Sigma\right)\deltat + \bigO(\deltat^2)\nonumber\\
		    &\xrightarrow[\deltat \to 0]{} V^\pi(s).
		    \label{eq:Q_cont}
\end{align}
A consequence of Eq.~\eqref{eq:Q_cont} is that in exact continuous time,
$Q^\pi$ does not bear any information on the hierarchy of actions, and
thus cannot be used to select actions that yields higher future returns.

As the difference between between $Q^\pi(s, a)$ and $V^\pi(s)$ is of order
$\bigO(\deltat)$, it is natural to define a rescaled version of the advantage
function, namely
\begin{align}
	A^\pi(s, a) &= \frac{Q^\pi(s,a) - V^\pi(s)}{\deltat}\\
		    &= \reward(s, a) +
		    \frac{\gamma^{\deltat} \E_{\delta s \sim \delta s(s, a)} V^\pi(s + \delta s) - V^\pi(s)}{\deltat}.
    \label{eq:adv}
\end{align}
Contrary to the action state value function, the rescaled advantage function converges
to an action dependent quantity when $\deltat$ goes to $0$
\begin{align}
	A^\pi(s, a) &\xrightarrow[\deltat \to 0]{} \reward(s, a) +
	(\gamma - 1) V^\pi(s) + \partial_s V^\pi(s) f(s, a)\nonumber\\
		    &+
	\frac{\text{tr}\left(\Sigma^T(s, a)\partial^2_{s^2}V^\pi(s) \Sigma(s,a)\right)}{2},
\end{align}
and $Q^\pi$ can then be rewritten as
\begin{equation}
	Q^\pi(s, a) = V^\pi(s) + \deltat A^\pi(s, a).
	\label{eq:reparam_q_pi}
\end{equation}

\section{Deep Advantage Updating}
In continuous time, $Q^\pi$ is independant of actions and can thus not be used
to select promising actions.  In near continuous time, $Q^\pi$ still depends on
actions, and can still be used to choose promising actions under the current
policy. However, when approximating $Q^\pi$, if the approximation error is much
bigger than $\bigO(\deltat)$, approximation errors dominates, and the hierarchy
of actions given by the approximation is likely to be erroneous.

When the approximate Q-function is initialized, if the effect of actions on the
Q-function is order of magnitudes higher than what it should be, approximating
$Q^\pi$ is likely to be difficult. Besides, the error is likely to be further
propagateed when the equation used to update our approximate $Q$ function
relies on bootstraping, as is the case for usual temporal difference derived methods.

\subsection{Reparameterizing $Q_\theta$}
\label{subsec:reparam}
To prevent overestimating the dependency on actions of the approximation of
$Q^\pi$, $Q_\theta$, in an environment with time discretization $\deltat$,
Eq.~\eqref{eq:reparam_q_pi} naturally yields the following reparameterization
\begin{equation}
	Q_\theta(s, a) = V_{\theta_1}(s) + \deltat A_{\theta_2}(s, a).
\end{equation}
To avoid cumbersome notations, in what follows, parameter indices are dropped.
The insight behind this parameterization is that when $Q$ correctly
approximates $Q^\pi$, $V$ should approximate the value function, while $A$
should approximate the advantage function, as defined in Eq.~\eqref{eq:adv}.


The proposed reparameterization does not, on its own, guarantee that
when $Q$ correctly approximates $Q^\pi$, $A$ approximates $A^\pi$.  Indeed, for
any given pair $V$, $A$, the transformed pair
\begin{align}
	\tilde{V}(s) &= V(s) - f(s)\\
	\tilde{A}(s, a) &= A(s, a) + \frac{f(s)}{\deltat}
\end{align}
yields the exact same $Q$ function. Consequently, $Q$ can correctly approximate
$Q^\pi$ even if $A$ is arbitrarily far from $A^\pi$.
To enforce, identifiability of $A$, one must enforce that 
\begin{equation}
	A(s, \pi(s)) = 0.
\end{equation}
With this additional constraint, if $Q^\pi = Q$,
\begin{align}
	A^\pi(s, a) &= \frac{Q^\pi(s,a) - V^\pi(s)}{\deltat}\\
		    &= \frac{Q(s,a) - Q(s, \pi(s))}{\deltat}\\
		    &= A(s, a).
\end{align}
In the spirit of~\cite{dueling_nets}, this condition is enforced by writing $A$ as
\begin{equation}
	A(s, a) = \bar{A}(s, a) - \bar{A}(s, \pi(s)).
\end{equation}

\subsection{Time invariant exploration}
\label{subsec:explo}
To obtain a time invariant reinforcement learning algorithm, a time invariant
exploration scheme is required. 

The exploration scheme used~\cite{ddpg}, which consists in adding an
\emph{Ornstein-Uhlenbeck}~\cite{orn-uhl} process to actions, is naturally time
invariant in the sense that it is first defined as a time continuous stochastic
process, and that discretization of this process are used in practice. Indeed,
a continuous Ornstein-Uhlenbeck process is defined as
\begin{equation}
	dx_t = - (x_t - \mu) \theta dt + \sigma \dbt
	\label{eq:orn_uhl}
\end{equation}
and is naturally discretized by turning $d$'s into $\delta$'s, and using
the same $\deltat$ for both the learning algorithm and the exploration scheme.
Consequently, this exploration scheme can readily be used in continuous action
environments.

On the other hand, the usual exploration scheme used in discrete actions
Q-learning, $\varepsilon$-greedy exploration~\cite{sutton} is not time
invariant, in the sense that resulting state trajectories do not converge to a
well defined stochastic process as $\deltat \rightarrow 0$. As $\dt$ goes to $0$,
for any $\varepsilon$ independant of $\dt$, the behavior of state trajectories get
closer to that of a white noise.

To obtain a temporally coherent exploration scheme in discrete action environment
when using Q-learning we instead directly apply a discretization of a time continuous
stochastic process to the output of the approximate Q-function, and select actions
by maximizing the resulting perturbed Q-function. Typically, the approximate Q-function
values are also perturbed using a discretized Ornstein-Uhlenbeck noise.

\subsection{Algorithms}
\label{subsec:algorithm}
Two variants of \emph{Deep Advantage Updating} are derived to cover both the continuous
and discrete action cases. They both rely on variants of Q-learning. More precisely,
we learn $V$ and $A$ using approximate version of the following bellman equation
\begin{align}
	V^{\pi}(s) + A^{\pi}(s, a) &= \E_{\delta r \sim \delta r(s, a)}\delta r\nonumber\\ 
				   &+ \gamma^{\deltat}  \E_{\delta s \sim \delta s(s, a)} V^{\pi}(s + \delta s)\label{eq:bellman_A}\\
	A^{\pi}(s, \pi(s)) &= 0\label{eq:max_A}
\end{align}
for a greedy exploitation policy $\pi(s) = \text{argmax}_{a'}A^\pi(s, a')$.
Eq.~\eqref{eq:max_A} is directly verified owing to the parameterization of $A$.
To approximately verify~\eqref{eq:bellman_A}, the corresponding squared residual
is minimized by an approximate gradient descent.
The corresponding update equations when learning from incoming transitions of
an exploratory trajectory are
\begin{align}
	\delta Q_t &= A(s_t, a_t) - \frac{\delta r_t + \gamma^{\deltat} V(s_{t+\deltat}) - V(s_t)}{\deltat}\\
	\theta^1_{t+\deltat} &= \theta^1_t + \eta_1 \partial_{\theta^1} V(s_t) \delta Q_t\\
	\theta^2_{t+\deltat} &= \theta^2_t + \eta_2 \partial_{\theta^2} A(s_t, a_t) \delta Q_t.
	\label{eq:approx_bellman_A}
\end{align}
where $\eta$'s are learning rates. We are now left to derive appropriate learning rates
to provide invariance to time discretization.
\begin{algorithm}[ht]
	\caption{Deep Advantage Updating (Discrete actions)}
	%! TEX root = icml_drau.tex
\begin{algorithmic}
	\STATE \textbf{Inputs:}
	\STATE $\theta$ and $\psi$, parameters of
	$V_{\theta}$ and $\bar{A}_{\psi}$.
	\STATE $\pi^{\text{explore}}$ and $\nu_\deltat$ defining an exploration policy.
	\STATE \textbf{opt}$_1$, \textbf{opt}$_2$, $\alpha_1 \deltat$ and $\alpha_2 \deltat$ optimizers and learning rates.
	\STATE $\mathcal{D}$, buffer of transitions $(s, a, r, d, s')$, with $d$ the episode termination signal.
	\STATE $\deltat$ and $\gamma$, time discretization and discount factor.
	\STATE \textbf{nb\_epochs} number of epochs.
	\STATE \textbf{nb\_steps}, number of steps per epoch.
	\STATE
	\STATE Observe initial state $s^0$
	\STATE $t \gets 0$
	\FOR {$e=0, \textbf{nb\_epochs}$}
	\FOR {$j=1, \textbf{nb\_steps}$}
	\STATE $a^k \leftarrow \pi^{\text{explore}}(s^k, \nu^k_\deltat)$.
	\STATE Perform $a^k$ and observe $(r^{k+1}, d^{k+1}, s^{k+1})$.
	\STATE Store $(s^k, a^k, r^{k+1}, d^{k+1}, s^{k+1})$ in $\mathcal{D}$.
	\STATE $k \gets k + 1$
	\ENDFOR
	\FOR {$k=0, \text{nb\_learn}$}
	\STATE \text{Sample a batch of $N$ random transitions from $\mathcal{D}$}
	\STATE $Q^i \gets V_{\theta}(s^i) + \deltat\hspace{-.17em}\left(
	\bar{A}_{\psi}(s^i, a^i) - \max\limits_{a'}\bar{A}_{\psi}(s^i, a')\right)$
	\STATE $\tilde{Q^i} \gets r^i\deltat + (1 - d^i) \gamma^{\deltat} V_{\theta}(s'^i)$
	\STATE $\Delta \theta \gets \frac{1}{N}\sum\limits_{i=1}^N  \frac{\left(Q^i - \tilde{Q^i}\right)\partial_{\theta} V_{\theta}(s^i)}{\deltat}$
	\STATE $\Delta \psi \gets \frac{1}{N}\sum\limits_{i=1}^N \frac{\left(Q^i - \tilde{Q^i}\right)\partial_{\psi} \left(\bar{A}_{\psi}(s^i, a^i) - \max\limits_{a'}\bar{A}_{\psi}(s^i, a')\right) }{\deltat}$
	\STATE Update $\theta$ with \textbf{opt}$_1$, $\Delta \theta$ and learning rate $\alpha_1 \deltat$.
	\STATE Update $\psi$ with \textbf{opt}$_2$, $\Delta \psi$ and learning rate $\alpha_2 \deltat$.
	\ENDFOR
	\ENDFOR
\end{algorithmic}

\end{algorithm}

\subsection{Learning rates scalings}
\label{subsec:lr}
A necessary condition to guarantee invariance to time discretization is that
the trajectory of the parameters of our function approximators should remain
well behaved when the discretization parameter $\deltat$ decreases. This notably
means that optimization steps should not be too large, in which case parameter
trajectories could diverge in a single gradient step, or two small, in which case
the trajectories would converge to single points as $\deltat$ goes to $0$.
This in turn imposes conditions on the scalings of the learning rates of invariant
algorithms with respect to the discretization timestep.

If we optimize $\theta_1$ and $\theta_2$ according to Eq.~\eqref{eq:approx_bellman_A},
both $\theta_1$ and $\theta_2$ verify update equations of the form
\begin{align}
	\
	\delta \theta^1_t = \eta_1 \partial_{\theta_1} V(s_t) (
\end{align}
To have parameter steps that do not grow larger or shrink to $0$ as $\deltat
\leftarrow 0$, both $\delta \theta^1$ and $\delta \theta^2$ should verify
stochastic differential equations, and notably have a deterministic component
of order $\deltat$. If the deterministic component 
\section{Experiments}
\label{sec:exp}

\section{Limitations}
\label{sec:limitations}
\section{Conclusion}
\label{sec:concl}

\bibliography{icml_drau.bib}
\bibliographystyle{icml2018}
\end{document}


% This document was modified from the file originally made available by
% Pat Langley and Andrea Danyluk for ICML-2K. This version was created
% by Iain Murray in 2018. It was modified from a version from Dan Roy in
% 2017, which was based on a version from Lise Getoor and Tobias
% Scheffer, which was slightly modified from the 2010 version by
% Thorsten Joachims & Johannes Fuernkranz, slightly modified from the
% 2009 version by Kiri Wagstaff and Sam Roweis's 2008 version, which is
% slightly modified from Prasad Tadepalli's 2007 version which is a
% lightly changed version of the previous year's version by Andrew
% Moore, which was in turn edited from those of Kristian Kersting and
% Codrina Lauth. Alex Smola contributed to the algorithmic style files.
