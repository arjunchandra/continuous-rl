%! TEX root = icml_drau.tex
\section{Related work}
\label{sec:related}

Our approach builds on \cite{adv_upd}, who identified the
collapse of Q-learning for small time steps and suggested the Advantage
Updating algorithm with proper $\deltat$ scalings as a solution; testing
was only done on a quadratic-linear problem. Our Deep Advantage Updating
expands this idea. Using a different normalization step for $A$ (which
forgoes the need to learn the normalization itself, thanks to the
parameterization \eqref{eq:Aparam}), we test Advantage Updating for the
first time on a variety on RL environments using deep networks,
establishing as a viable algorithm in this setting.
Additionally, we provide formal proofs for the collapse
of Q-learning when $\deltat\to 0$, and for the non-collapse of Advantage
Updating.

\TODO{Say that Actor-critic has no problem when $\deltat\to 0$?}

Continuous and near continuous reinforcement learning, which is at the core of
our approach has been thoroughly studied in the deterministic case
in~\cite{adv_upd} or \cite{cont_rl}, but has not been applied in the context of
deep reinforcement learning.

The approach presented in this paper is closely related to \cite{adv_upd} and
\cite{cont_rl}. \cite{adv_upd} already separated the learning procedure of the
value and advantage functions, and introduced proper scalings between the two
components. We extend on \cite{adv_upd} in three directions. We provide a
formal argument on why the advantage contribution of the Q-function is of order
$\bigO(\deltat)$, irrelevant of the stochasticity of the environment, in a quite
general framework. We show that, provided with well suited learning rates and
exploration method, Advantage updating provide near invariance to change of time
discretization. We show that Advantage updating is viable in the context of deep
reinforcement learning, while not using the true gradient of the residual error.

More recently, \cite{dueling_nets} also introduced a parameterization separating
the value and the advantage components of the Q-function. We additionally
introduce a natural scaling coeficient between the two, that strongly improves
the time discretization invariance properties.

