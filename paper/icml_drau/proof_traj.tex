%! TEX root = icml_drau.tex
For any $k$, define
\begin{equation}
	e_\deltat^k = \|s_\deltat^k - s_{\deltat k}\|.
\end{equation}
Taylor integral formula gives
\begin{align}
	s_\deltat^{k + 1} &= s_\deltat^k + F(s_\deltat^k, \pi(s_\deltat^k)) \deltat +
	\frac{1}{2}\int\limits_{0}^\deltat (\deltat - t) \frac{d^2 \tilde{s}_t}{dt^2} dt\\
	s_{\deltat(k + 1)} &= s_{\deltat k} + F(s_{\deltat k}, \pi(s_{\deltat k})) \deltat +
	\frac{1}{2}\int\limits_0^\deltat (\deltat - t) \frac{d^2 s_{t + \deltat k}}{dt^2} dt.
\end{align}
Now, both $d^2 s_t/dt^2$ and $d^2 \tilde{s}_t/dt^2$ are uniformly bounded, by
boundedness and lipschitzianity of $s \rightarrow F(s, \pi(s))$ and $s
\rightarrow F(s, \pi(s_\deltat^k))$. Consequently, there exists $C$ such that
\begin{align}
	e_\deltat^{k + 1} &\leq \|s_\deltat^k - s_{\deltat k}\| + \|F(s_\deltat^k, \pi(s_\deltat^k)) - F(s_{\deltat k}, \pi(s_{\deltat k}))\| \deltat + C \deltat^2\\
			  &\leq (1 + K \deltat) e_\deltat^k + C \deltat^2.
\end{align}
Now, it is easy to prove by recurrence that
\begin{equation}
	e_\deltat^k \leq (1 + K \deltat)^k (e_\deltat^0 + \frac{C}{K} \deltat) - \frac{C}{K}\deltat.
\end{equation}
As $e_\deltat^0 = 0$, this translates to
\begin{align}
	e_\deltat^k &\leq ((1 + K \deltat)^k - 1) \deltat \frac{C}{K}\\
		    &\leq (e^{K \deltat k} - 1) \deltat \frac{C}{K}.
\end{align}
Consequently,
\begin{equation}
	e_\deltat^\lfloor t / \deltat \rfloor \leq (e^{K(t + \deltat)} - 1) \deltat \frac{C}{K}
\end{equation}
