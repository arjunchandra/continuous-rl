%! TEX root = supplementary.tex
All the details specifying our implementation are hereby given. We first give precise pseudo code
descriptions for both \emph{Continuous Deep Advantage Updating} (Alg.~\ref{alg:cdau}), as well as the variants of DDPG (Alg.~\ref{alg:ddpg})
and DQN (Alg.~\ref{alg:dqn}) used.

\begin{algorithm}
	%! TEX root = supplementary.tex

\begin{algorithmic}
	\STATE \textbf{Inputs:}
	\STATE $\theta$, $\psi$ and $\phi$, parameters of
	$V_{\theta}$, $\bar{A}_{\psi}$ and $\pi_\phi$.
	\STATE $\pi^{\text{explore}}$ and $\nu_\deltat$ defining an exploration policy.
	\STATE \textbf{opt}$_V$, \textbf{opt}$_A$, \textbf{opt}$_\pi$, $\alpha^V \deltat$, $\alpha^A \deltat$ and $\alpha^\pi\deltat$, optimizers and learning rates.
	\STATE $\mathcal{D}$, buffer of transitions $(s, a, r, d, s')$, with $d$ the episode termination signal.
	\STATE $\deltat$ and $\gamma$, time discretization and discount factor.
	\STATE \textbf{nb\_epochs} number of epochs.
	\STATE \textbf{nb\_steps}, number of steps per epoch.
	\STATE
	\STATE Observe initial state $s^0$
	\STATE $t \gets 0$
	\FOR {$e=0, \textbf{nb\_epochs}$}
	\FOR {$j=1, \textbf{nb\_steps}$}
	\STATE $a^k \leftarrow \pi^{\text{explore}}(s^k, \nu^k_\deltat)$.
	\STATE Perform $a^k$ and observe $(r^{k+1}, d^{k+1}, s^{k+1})$.
	\STATE Store $(s^k, a^k, r^{k+1}, d^{k+1}, s^{k+1})$ in $\mathcal{D}$.
	\STATE $k \gets k + 1$
	\ENDFOR
	\FOR {$k=0, \text{nb\_learn}$}
	\STATE \text{Sample a batch of $N$ random transitions from $\mathcal{D}$}
	\STATE $Q^i \gets V_{\theta}(s^i) + \deltat\hspace{-.17em}\left(
	\bar{A}_{\psi}(s^i, a^i) - \bar{A}_{\psi}(s^i, \pi_\phi(s^i))\right)$
	\STATE $\tilde{Q^i} \gets r^i\deltat + (1 - d^i) \gamma^{\deltat} V_{\theta}(s'^i)$
	\STATE $\Delta \theta \gets \frac{1}{N}\sum\limits_{i=1}^N  \frac{\left(Q^i - \tilde{Q^i}\right)\partial_{\theta} V_{\theta}(s^i)}{\deltat}$
	\STATE $\Delta \psi \gets \frac{1}{N}\sum\limits_{i=1}^N \frac{\left(Q^i - \tilde{Q^i}\right)\partial_{\psi} \left(\bar{A}_{\psi}(s^i, a^i) - \bar{A}_{\psi}(s^i, \pi_\phi(s^i))\right) }{\deltat}$
	\STATE $\Delta \phi \gets \frac{1}{N} \sum\limits_{i=1}^N \partial_a \bar{A}_\psi(s^i, \pi_\phi(s^i)) \partial_\phi \pi_\phi(s^i)$
	\STATE Update $\theta$ with \textbf{opt}$_V$, $\Delta \theta$ and learning rate $\alpha^V \deltat$.
	\STATE Update $\psi$ with \textbf{opt}$_A$, $\Delta \psi$ and learning rate $\alpha^A \deltat$.
	\STATE Update $\phi$ with \textbf{opt}$_\pi$, $\Delta \phi$ and learning rate $\alpha^\pi \deltat$.
	\ENDFOR
	\ENDFOR
\end{algorithmic}

	\caption{Continuous DAU}
	\label{alg:cdau}
\end{algorithm}
\begin{algorithm}
	%! TEX root = supplementary.tex

\begin{algorithmic}
	\STATE \textbf{Inputs:}
	\STATE $\psi$ and $\phi$, parameters of
	$Q_\psi$ and $\pi_\phi$.
	\STATE $\psi'$ and $\phi'$, parameters of target networks
	$Q_{\psi'}$ and $\pi_{\phi'}$.
	\STATE $\pi^{\text{explore}}$ and $\nu$ defining an exploration policy.
	\STATE \textbf{opt}$_Q$, \textbf{opt}$_\pi$, $\alpha^Q$ and $\alpha^\pi$, optimizers and learning rates.
	\STATE $\mathcal{D}$, buffer of transitions $(s, a, r, d, s')$, with $d$ the episode termination signal.
	\STATE $\gamma$ discount factor.
	\STATE $\tau$ target network update factor.
	\STATE \textbf{nb\_epochs} number of epochs.
	\STATE \textbf{nb\_steps}, number of steps per epoch.
	\STATE
	\STATE Observe initial state $s^0$
	\STATE $t \gets 0$
	\FOR {$e=0, \textbf{nb\_epochs}$}
	\FOR {$j=1, \textbf{nb\_steps}$}
	\STATE $a^k \leftarrow \pi^{\text{explore}}(s^k, \nu^k)$.
	\STATE Perform $a^k$ and observe $(r^{k+1}, d^{k+1}, s^{k+1})$.
	\STATE Store $(s^k, a^k, r^{k+1}, d^{k+1}, s^{k+1})$ in $\mathcal{D}$.
	\STATE $k \gets k + 1$
	\ENDFOR
	\FOR {$k=0, \text{nb\_learn}$}
	\STATE \text{Sample a batch of $N$ random transitions from $\mathcal{D}$}
	\STATE $\tilde{Q^i} \gets r^i + (1 - d^i) \gamma Q_{\psi'}(s'^i, \pi_{\phi'}(s'^i))$
	\STATE $\Delta \psi \gets \frac{1}{N}\sum\limits_{i=1}^N \left(Q^i - \tilde{Q^i}\right) \partial_\psi Q(s^i, a^i)$
	\STATE $\Delta \phi \gets \frac{1}{N} \sum\limits_{i=1}^N \partial_a Q_\psi(s^i, \pi_\phi(s^i)) \partial_\phi \pi_\phi(s^i)$
	\STATE Update $\psi$ with \textbf{opt}$_Q$, $\Delta \psi$ and learning rate $\alpha^Q$.
	\STATE Update $\phi$ with \textbf{opt}$_\pi$, $\Delta \phi$ and learning rate $\alpha^\pi$.
	\STATE $\psi' \gets \tau \psi' + (1 - \tau) \psi$
	\STATE $\phi' \gets \tau \phi' + (1 - \tau) \phi$
	\ENDFOR
	\ENDFOR
\end{algorithmic}

	\caption{DDPG}
	\label{alg:ddpg}
\end{algorithm}
\begin{algorithm}
	%! TEX root = supplementary.tex

\begin{algorithmic}
	\STATE \textbf{Inputs:}
	\STATE $\psi$ parameter of
	$Q_\psi$.
	\STATE $\psi'$, parameters of target networks
	$Q_{\psi'}$.
	\STATE $\pi^{\text{explore}}$ and $\nu$ defining an exploration policy.
	\STATE \textbf{opt}$_Q$, $\alpha^Q$ optimizer and learning rate.
	\STATE $\mathcal{D}$, buffer of transitions $(s, a, r, d, s')$, with $d$ the episode termination signal.
	\STATE $\gamma$ discount factor.
	\STATE $\tau$ target network update factor.
	\STATE \textbf{nb\_epochs} number of epochs.
	\STATE \textbf{nb\_steps}, number of steps per epoch.
	\STATE
	\STATE Observe initial state $s^0$
	\STATE $t \gets 0$
	\FOR {$e=0, \textbf{nb\_epochs}$}
	\FOR {$j=1, \textbf{nb\_steps}$}
	\STATE $a^k \leftarrow \pi^{\text{explore}}(s^k, \nu^k)$.
	\STATE Perform $a^k$ and observe $(r^{k+1}, d^{k+1}, s^{k+1})$.
	\STATE Store $(s^k, a^k, r^{k+1}, d^{k+1}, s^{k+1})$ in $\mathcal{D}$.
	\STATE $k \gets k + 1$
	\ENDFOR
	\FOR {$k=0, \text{nb\_learn}$}
	\STATE \text{Sample a batch of $N$ random transitions from $\mathcal{D}$}
	\STATE $\tilde{Q^i} \gets r^i + (1 - d^i) \gamma \max\limits_{a'}Q_{\psi'}(s'^i, a')$
	\STATE $\Delta \psi \gets \frac{1}{N}\sum\limits_{i=1}^N \left(Q^i - \tilde{Q^i}\right) \partial_\psi Q(s^i, a^i)$
	\STATE Update $\psi$ with \textbf{opt}$_Q$, $\Delta \psi$ and learning rate $\alpha^Q$.
	\STATE $\psi' \gets \tau \psi' + (1 - \tau) \psi$
	\ENDFOR
	\ENDFOR
\end{algorithmic}

	\caption{DQN}
	\label{alg:dqn}
\end{algorithm}

For DDPG and DQN, two different settings were experimented with:
\begin{itemize}
	\item One with time discretization scalings, to keep the comparison
		fair. In this setting, the discount factor is still scaled as $\gamma^\deltat$,
		rewards are scaled as $r \deltat$, and learning rates are scaled to obtain parameter
		updates of order $\deltat$. As RMSprop is used for all experiments, this amounts
		to using a learning rate scaling as $\alpha^Q = \tilde{\alpha}^Q \deltat$,
		$\alpha^\pi = \tilde{\alpha}^\pi \deltat$.
	\item One without discretization scalings. In that case, only the discount factor is scaled
		as $\gamma^\deltat$, to prevent unfair shortsightedness. All other
		parameters are set with a reference $\deltat_0 = 1e-2$. For instance,
		for all $\deltat$'s, the reward perceived is $r * \deltat_0$, and
		similarily for learning rates, $\alpha^Q = \tilde{\alpha}^Q
		\deltat_0$, $\alpha^\pi = \tilde{\alpha}^Q \deltat_0$. These scalings
		don't depend on the discretization, but perform decently at least for
		the highest discretization.
\end{itemize}
\subsection{Global hyperparameters}
The following hyperparameters are maintained constant throughout all our experiments,
\begin{itemize}
	\item All networks used are of the form
		\begin{verbatim}
		Sequential(
		    Linear(nb_inputs, 256),
		    LayerNorm(256),
		    ReLU(),
		    Linear(256, 256),
		    LayerNorm(256),
		    ReLU(),
		    Linear(256, nb_outputs)
		).
		\end{verbatim}
	Policy networks have an additional $\tanh$ layer to constraint action range. On certain
	environments, network inputs are normalized by applying a mean-std normalization, with
	mean and standard deviations computed on each individual input features, on all previously
	encountered samples.
	\item $\mathcal{D}$ is a cyclic buffer of size $1000000$.
	\item $\textbf{nb\_steps}$ is set to $10$, and $256$ environments are run in parallel to
		accelerate the training procedure, totalling $2560$ environment interactions between
		learning steps.
	\item $\textbf{nb\_learn}$ is set to $50$.
	\item The physical $\gamma$ is set to $0.8$. It is always scaled as $\gamma^\deltat$ (even for
		unscaled DQN and DDPG).
	\item $N$, the batch size is set to $256$.
	\item RMSprop is used as an optimizer without momentum, and with
		$\alpha=1 - \deltat$ (or $1 - \deltat_0$ for unscaled DDPG and
		DQN).
	\item Exploration is always performed as described in the main text. The OU process used as
		parameters $\kappa = 7.5$, $\sigma = 1.5$.
	\item Unless otherwise stated, $\alpha_1 \deq \tilde{\alpha}^Q = \alpha^V = \alpha^A = 0.1$, $\alpha_2 \deq \tilde{\alpha}^\pi =
		\alpha^\pi = 0.03$.
	\item $\tau = 0.9$
\end{itemize}
\subsection{Environment dependent hyperparameters}
We hereby list the hyperparameters used for each environment. Continuous actions environments are marked with a
(C), discrete actions environments with a (D).
\begin{itemize}
	\item {\bf Ant (C)}: State normalization is used. Discretization range: $[0.05, 0.02, 0.01, 0.005, 0.002]$.
	\item {\bf Cheetah (C)}: State normalization is used. Discretization range: $[0.05, 0.02, 0.01, 0.005, 0.002]$
	\item {\bf Bipedal Walker (C)\footnote{
				The reward for Bipedal Walker is modified not to scale with $\deltat$. This does not introduce any change for the default setup.
		}}: State normalization is used, $\alpha_2 = 0.02$. Discretization range: $[0.01, 0.005, 0.002, 0.001]$.
	\item {\bf Cartpole (D)}: $\alpha_2 = 0.02$, $\tau = 0$. Discretization range: $[0.01, 0.005, 0.002, 0.001, 0.0005]$.
	\item {\bf Pendulum (C)}: $\alpha_2 = 0.02$, $\tau = 0$. Discretization range: $[0.01, 0.005, 0.002, 0.001, 0.0005]$.

\end{itemize}
