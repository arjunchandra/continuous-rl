\documentclass[11pt]{article}

\usepackage{amsmath}
\usepackage{amssymb}
\usepackage{amsthm}
\usepackage{mathtools}
\usepackage{xcolor}

\newtheorem{theorem}{Theorem}
%! TEX root = main.tex
\usepackage{xcolor}
\usepackage{algorithm}
\usepackage{algpseudocode}
\usepackage{hyperref}

%\newcommand{\deltat}{\delta\hspace*{-.3mm}t}
\newcommand{\deltat}{\ensuremath{\delta\hspace{-.06em}t}}
\newcommand{\bigO}[1]{O (#1)}
\newcommand{\reward}{\tilde{r}}
\newcommand{\actionspace}{\mathcal{A}}
\newcommand{\statespace}{\mathcal{S}}
\newcommand{\NDC}[1]{\color{red}NDC:#1\color{black}}

\newcommand{\pfield}{\mathcal{F}_t}
\newcommand{\Ep}[1]{\BE\left[#1\mid\pfield\right]}
\newcommand{\diag}[1]{\mathrm{\textbf{diag}}(#1)}
\newcommand{\lmax}{\lambda_{\mathrm{max}}}



\newcommand{\mynumero}{n°}


%\def\d{{\mathrm{d}}}
%\def\d{\operatorname{d}\!}
\def\d{\operatorname{d}\!{}}
%\def\d{\operatorname{}\!\mathrm{d}}

\def\N{{\mathbb{N}}}
\def\Z{{\mathbb{Z}}}
\def\Nstar{{\mathbb{N}^\star}}
\def\Q{{\mathbb{Q}}}
\def\R{{\mathbb{R}}}
\def\C{{\mathbb{C}}}

\renewcommand{\geq}{\geqslant}
\renewcommand{\leq}{\leqslant}

\renewcommand{\emptyset}{\varnothing}

\newcommand{\deq}{\mathrel{\mathop{:}}=}
\newcommand{\eqd}{=\mathrel{\mathop{:}}}

\newcommand{\from}{\colon} % correct ':' in f\from X \to Y
\newcommand{\st}{\mid} % set builder
%\newcommand{\st}{\mathrel{}\middle|\mathrel{}} % set builder

\def\eps{\varepsilon}
\renewcommand{\epsilon}{\varepsilon}
\renewcommand{\phi}{\varphi}

\def\ds{\displaystyle}

\DeclareMathOperator{\dist}{dist}
\DeclareMathOperator{\diam}{diam}
\DeclareMathOperator{\vol}{vol}
\DeclareMathOperator{\Ric}{Ric}

\DeclareMathOperator{\lap}{\Delta\!}
\DeclareMathOperator{\nab}{\nabla\!\!}
\DeclareMathOperator{\Hess}{Hess}

\DeclareMathOperator{\Ent}{Ent}
\DeclareMathOperator{\Var}{Var}
\DeclareMathOperator{\Cov}{Cov}
\let\oldPr\Pr
\renewcommand{\Pr}{\oldPr\nolimits}
\newcommand{\E}{\mathbb{E}}
\newcommand{\KL}[2]{\mathrm{KL}\!\left(#1 \,|\hspace{-.15ex}|\,#2\right)}

\DeclareMathOperator{\mult}{mult}
\DeclareMathOperator{\Card}{Card}
\DeclareMathOperator{\Aut}{Aut}
\DeclareMathOperator{\Epi}{Epi}
\DeclareMathOperator{\Spec}{Sp}
%\DeclareMathOperator{\Ker}{Ker}
\DeclareMathOperator{\Img}{Im}
\DeclareMathOperator{\Tr}{Tr}
\DeclareMathOperator{\tr}{tr}
\DeclareMathOperator{\Tor}{Tor}
\DeclareMathOperator{\Ext}{Ext}
\DeclareMathOperator{\Hom}{Hom}
\DeclareMathOperator{\End}{End}
\DeclareMathOperator{\coker}{coker}
\DeclareMathOperator{\Id}{Id}
\DeclareMathOperator{\id}{id}
%\DeclareMathOperator{\diag}{diag}


\newcommand{\abs}[1]{\left\lvert#1\right\rvert}
\newcommand{\norm}[1]{\left\lVert#1\right\rVert}
\newcommand{\scal}[2]{\left< \, #1 \mid #2 \, \right>}
\newcommand{\1}{\mathbbm{1}}

\newcommand{\ilim}[1]{\underset{#1}{\underrightarrow{\lim\vspace{.5ex}}}\,}
\newcommand{\plim}[1]{\underset{#1}{\underleftarrow{\lim\vspace{.5ex}}}\,}

\DeclareMathOperator*{\vlimsup}{\varlimsup}
\DeclareMathOperator*{\vliminf}{\varliminf}

\newcommand{\presgroup}[2]{\left\langle\,#1 \mid  #2\,\right\rangle}

\newcommand{\twopi}{2\hspace{-.23em}\pi}

%\DeclareMathOperator*{\argmax}{arg\,max}
%\DeclareMathOperator*{\argmin}{arg\,min}



\begin{document}
\section{Proofs}
We hereby give proofs for all the results presented in the paper.
The first result presented is a proof of convergence for discretized
trajectories.
\begin{theorem}
	Let $F: {\cal S} \times {\cal A} \rightarrow \bb{R}^n$ and $\pi: {\cal S}
	\rightarrow {\cal A}$ be the dynamic and policy functions. Assume that,
	for any $a$, $s \rightarrow F(s, a)$ and $s \rightarrow F(s, \pi(s))$
	are ${\cal C}^1$, bounded and $K$-lipschitz.  For
	a given $s_0$, define the trajectory $(s_t)_{t\geq 0}$ as the unique
	solution of the differential equation
	\begin{equation}
		\frac{ds_t}{dt} = F(s_t, \pi(s_t)).
		\label{eq:diff}
	\end{equation}
	For any $\deltat > 0$, define the discretized trajectory
	$(s_\deltat^k)_k$ recurrently as $s_\deltat^0 = s_0$,
	$s_\deltat^{k + 1}$ is the value at time $\deltat$ of
	the unique solution of
	\begin{equation}
		\frac{d\tilde{s}_t}{dt} = F(\tilde{s}_t, \pi(s_\deltat^k))
	\end{equation}
	with initial point $s_\deltat^k$.
	Then, there exists $C > 0$ such that, for every $t \geq 0$
	\begin{equation}
		\|s_t - s_\deltat^{\lfloor \frac{t}{\deltat} \rfloor}\|
		\leq \deltat \frac{C}{K}e^{Kt}.
	\end{equation}
	Notably, discretized trajectories converge pointwise to continuous trajectories.
	\label{th:traj-conv}
\end{theorem}
\begin{proof}
	%! TEX root = icml_drau.tex

\end{proof}
In what follows, we will assume that the reward function $r: {\cal S} \times {\cal A} \rightarrow \bb{R}$
is bounded, to ensure existence of $V^\pi$ and $V^\pi_\deltat$ for all $\deltat$.\begin{theorem}
	Under suitable assumptions \TODO{ref appendix}, for all $s \in {\cal
	S}$, one has
	%\begin{equation}
	%\label{eq:conv-value}
	$V^\pi_\deltat(s) = V^\pi(s) + \bigO(\deltat)$
	%\end{equation}
	when $\deltat\to 0$.
	\label{th:conv-value}
\end{theorem}
\begin{proof}
	%! TEX root = icml_drau.tex


\end{proof}

For the following proof, we further assume that both $V^\pi$ and
$V^\pi_\deltat$ are continuously differentiable, and that the gradient and
hessian of $V^\pi_\deltat$ w.r.t. $s$ are uniformly bounded in both $s$ and $\deltat$.
We also assume convergence of $\partial_s V^\pi_\deltat(s)$ to $\partial_s V^\pi(s)$ for
all $s$.
\begin{theorem}
	Under the above hypothesis, there exists $A^\pi: {\cal S} \rightarrow
	\bb{R}$ such that $A^\pi_\deltat$ converges pointwise to $A^\pi$ as
	$\deltat$ goes to $0$.
\end{theorem}
\begin{proof}
	%! TEX root = icml_drau.tex


\end{proof}

We now show that policy improvement works with the continous time advantage function, i.e.\
\begin{theorem}
	Let $\pi$ and $\pi'$ be two policies such that both $s \rightarrow r(s, \pi(s))$ and
	$s \rightarrow r(s, \pi'(s))$ are continuous.
	Assume that both $V^\pi$ and $V^{\pi'}$ are continuously differentiable.
	Define the advantage function for policies $\pi$ and $\pi'$ as in Eq.~\eqref{eq:adv_function}.

	If for all $s$, $A^\pi(s, \pi'(s)) \geq 0$, then for all $s$, $V^\pi(s) \leq V^{\pi'}(s)$.
\end{theorem}
\begin{proof}
	%! TEX root = supplementary.tex
Let $(s_t)_{t\geq 0}$ be a trajectory sampled from $\pi'$ i.e.\ solution of the equation
\begin{equation}
	ds_t / dt = F(s_t, \pi'(s_t))
\end{equation}
with initial condition $s_0 = s$.

Define
\begin{equation}
	B(T) = \int_{t=0}^T \gamma^t r(s_t, \pi'(s_t)) dt + \gamma^T V^\pi(s_T).
\end{equation}
This function if continuously differentiable, and its derivative is
\begin{align}
	\dot{B}(T) &= \gamma^T r(s_T, \pi'(s_T)) + \gamma^T \partial_s V^\pi(s) F(s, \pi'(s)) + \gamma^T \ln(\gamma) V^\pi(s_T)\\
		   &= \gamma^T A^\pi(s_T, \pi'(s))
		   &\geq 0.
\end{align}
Thus $B$ is increasing, and $B(0) = V^\pi(s)$, $\lim\limits_{T\rightarrow \infty} B(t) = V^{\pi'}(s)$.
Consequently, $V^\pi(s) \leq V^{\pi'}(s)$, which ends the proof.


\end{proof}
\begin{theorem}
	Discrete parameter trajectories converge to continuous trajectories.
\end{theorem}
\begin{proof}
	%! TEX root = icml_drau.tex


\end{proof}
\section{Implementation details}
\end{document}
