%! TEX root = icml_drau.tex
\section{Discussion}
\label{sec:discussions}

The method derived in this work seems to yield improved robustness to
time discretization on various environments, e.g. pendulum swing up, or
simple locomotion tasks.  However on certain environments, there is still
room for improvment. In what follows, we identify some of the issues
that could explain this theoretical/practical discrepancy.

\paragraph{Regularity of the value function}~In all of our proofs, $V^\pi$ is
assumed to be continuously differentiable for all $\pi$.  This is a strong
hypothesis that is not always verified in practice. Very simple environments
break this hypothesis, such as for instance, the pendulum swing up environment,
with pendulum height as reward and torque as action.  Indeed, assume that $\pi$
is the optimal policy for the pendulum swing up environment, with state
representation $(\theta, \dot{\theta})$. The state space for this environment
can be partitioned into two subspaces $\mathcal{G}$ and $\mathcal{B}$, with
$\mathcal{G}$ the subspace where when following the optimal policy, the
pendulum can be swang up without crossing the $y$-axis, and $\mathcal{B}$ where
it have to cross the $y$-axis. The difference of V-function between a point in
$\mathcal{G}$ and a point in $\mathcal{B}$ is lower bounded by a finite
quantity $M$. Besides, one can find two points $s_1 \in \mathcal{G}$ and $s_2 \in \mathcal{B}$
such that $s_1$ and $s_2$ are arbitrarily close. Consequently, $V^\pi$ is
discontinuous on the frontier, which results in non infinitesimal advantages
for actions taken on the frontier. \TODO{Non infinitesimal advantages are found,
	more generally, as soon as a given policy has different regimes, depending on
	where it starts in state space. For such policies, at the frontier between two regimes,
the value function can be discontinuous, resulting in non infinitesimal advantages.}

\paragraph{Memory buffer size scalings}~Thm.~\ref{th:cont-params} is stated for
transitions sampled sequentially from a fixed infinite size sequence. In practice,
transitions are sampled from a memory buffer, to prevent excessive correlations.
The memory buffer used is a fixed size circular buffer, and is
filled with samples coming from a single infinite exploratory trajectory. In
our experiments, the same buffer size is used for all time discretization. When
$\deltat$ goes to $0$, this corresponds to learning nearly online on the
infinite sequence, since the buffer will only contain examples that are very
close temporally to the end of the sequence in physical time. While the resulting
algorithm \TODO{kind of} converges to a continuous time limit, it is still not
time discretization invariant, as the freshness of samples in the buffer will
vary with the time discretization.
A different choice would be to scale the size of the buffer by a factor of
$\frac{1}{\deltat}$. This would amount to always learning on the same physical time interval of
the trajectory. Our analysis, however, does not cover this case.

\paragraph{Near continuous time reinforcement learning and RMSprop}~RMSprop~\cite{rmsprop}
estimates a batched version of the second momentum of gradients using moving
averages, and divides gradient steps by the square root of this estimate. In
deterministic environments, gradient noise typically scales as $\bigO(1)$ in
term of $\deltat$, as seen in Eq.~\eqref{eq:approx_bellman_A}.  Consequently, RMSprop
preconditioning have no effect on the order of magnitude of required learning
rates. However, in near continuous stochastic environments, as in
Eq.~\ref{eq:sde}, variance of the gradient noise typically scales as
$\bigO\left(\frac{1}{\deltat}\right)$. If batch size does not depend on
$\deltat$, RMSprop preconditioning thus multiplies gradients by a factor $\bigO(\sqrt{\deltat})$,
and learning rates need only be scaled as $\alpha \sqrt{\deltat}$ instead of
$\alpha \deltat$.
