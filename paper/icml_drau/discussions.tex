%! TEX root = icml_drau.tex
\section{Discussion}
\label{sec:discussions}

The method derived in this work seems to yield improved robustness to
time discretization on various environments, e.g. pendulum swing up, or
simple locomotion tasks.  However on certain environments, there is still
room for improvment. In what follows, we identify some of the issues
that could explain this theoretical/practical discrepancy.

\paragraph{Smoothness of the value function.} In our proofs, $V^\pi$ is
assumed to be continuously differentiable.  This %is a strong
hypothesis is not always satisfied in practice. 
For instance, in the pendulum swing-up environment, depending on initial
position and momentum, the optimal policy may need to oscillate before
reaching the target state. The optimal value
function is discontinuous at the boundary between states where
oscillations are needed and those where they are not.
% Very simple environments
% break this hypothesis, such as the pendulum swing-up environment,
% with pendulum height as reward and torque as action.  Indeed, 
% % assume that $\pi$
% % is the optimal policy for the pendulum swing-up environment, with state
% % representation $(\theta, \dot{\theta})$. 
% the state space $(\theta, \dot{\theta})$ for this environment
% can be partitioned into two subspaces $\mathcal{G}$ and $\mathcal{B}$, with
% $\mathcal{G}$ the subspace where the optimal policy 
% can swing up the pendulum without crossing the $y$-axis, and $\mathcal{B}$ where
% it has to cross the $y$-axis. The difference of $V$-function between a point in
% $\mathcal{G}$ and a point in $\mathcal{B}$ is lower bounded by a finite
% quantity. 
% %Besides, one can find two points $s_1 \in \mathcal{G}$ and $s_2 \in \mathcal{B}$
% %such that $s_1$ and $s_2$ are arbitrarily close.
% Consequently, the value function is
%discontinuous on the frontier between $\mathcal{G}$ and $\mathcal{B}$, which 
This results in non infinitesimal advantages
for actions on the boundary. In such environments where a given policy
has different regimes depending on the initial state, the continuous-time
analysis would only hold almost-everywhere.
% more generally, as soon as a given policy has different regimes depending on
% where it starts in state space. For such policies the value
% function is typically discontinuous at the frontier between two
% regimes, resulting in non-infinitesimal advantages. The
% continuous-time DAU properties would only hold almost-everywhere.

\paragraph{Memory buffer size scalings}~Thm.~\ref{th:cont-params} is stated for
transitions sampled sequentially from a fixed trajectory. In practice,
transitions are sampled from a memory buffer, to prevent excessive correlations.
We used a fixed size circular buffer, filled
with samples from a single growing exploratory trajectory. In
our experiments, the same buffer size is used for all time discretizations. 
Thus the physical-time freshness of samples in the buffer varies with the
time discretization, and in the strictest sense using a fixed-size buffer
breaks timestep invariance. A memory-intensive option would be to use a
buffer of size $\frac{1}{\deltat}$ (fixed physical time); our theoretical
analysis does not cover this case.
% When
% $\deltat$ goes to $0$, this corresponds to learning nearly online on the
% infinite sequence, since the buffer will only contain examples that are very
% close temporally to the end of the sequence in physical time. \TODO{this
% contradicts the use of the buffer against correlations} While the resulting
% algorithm converges to a continuous time limit, in the strictest sense it is not
% time discretization invariant, as the freshness of samples in the buffer will
% vary with the time discretization.
% A different choice would be to scale the size of the buffer by a factor of
% $\frac{1}{\deltat}$. This would amount to always learning on the same physical time interval of
% the trajectory. Our theoretical analysis does not cover this case.
% \TODO{a bit too many details here?}

\paragraph{Near-continuous reinforcement learning and RMSProp.}
RMSProp~\cite{rmsprop} divides gradient steps by the square root of a
moving average estimate of the second moment of gradients.
This may interact with the learning rate scaling discussed above. In
deterministic environments, gradients typically scale as $\bigO(1)$ in
term of $\deltat$, as seen in \eqref{eq:approx_bellman_A}.  In that case, RMSProp
preconditioning has no effect on the suitable order of magnitude for learning
rates. However, in near continuous \emph{stochastic} environments (Eq.~\ref{eq:sde}), variance of the gradients typically scales as
$\bigO\left(\frac{1}{\deltat}\right)$. With a fixed batch size,
RMSProp will multiply gradients by a factor $\bigO(\sqrt{\deltat})$. In
that case,
learning rates need only be scaled as $\alpha \,\sqrt{\deltat}$ instead of
$\alpha \,\deltat$.

More generally, the continuous-time analysis should in principle be repeated for every component of  
a system. For instance, if a recurrent model is used to handle state
memory or partial observability, care should be taken that the model is
able to maintain memory for a non-infinitesimal physical-time when
$\deltat\to 0$ (see e.g.~\citealt{chronornn}).
