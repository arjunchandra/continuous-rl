%! TEX root = supplementary.tex
Let $(s_t, a_t)_{t\geq 0}$ be the trajectory on which parameters are
learnt. To simplify notations, define
\begin{equation}
	A_\psi(s, a) = \bar{A}_\psi(s, a) - \bar{A}_\psi(s, \pi(s)).
\end{equation}
Define $F$ as
\begin{align}
	F^\theta(\theta, \psi, s, a) &= \alpha^V (r(s, a) + \ln(\gamma) V_\theta(s) + \partial_s V_\theta(s) F(s, a) - A_\psi(s, a))\partial_\theta V_\theta(s)\\
	F^\psi(\theta, \psi, s, a) &= \alpha^A (r(s, a) + \ln(\gamma) V_\theta(s) + \partial_s V_\theta(s) F(s, a) - A_\psi(s, a))\partial_\psi A_\psi(s, a).
\end{align}
Both $V$ and $A$ are uniformly Lipschitz continuous in $\theta$ and $\psi$, thus
$F$ is Lipschitz continuous. \TODO{need 1 more derivative assumption on
$V, A$,
because $F$ uses derivatives of $V, A$.}

The discrete equations for parameters updates with learning rates $\alpha^V \deltat^\beta$ and
$\alpha^A \deltat^\beta$ are
\begin{align}
	\delta Q &= r(s_{k \deltat}, a_{k \deltat}) \deltat + 
		\gamma^\deltat V_{\theta^{k}_\deltat}(s_{(k + 1)\deltat}) - 
		V_{\theta^k_\deltat}(s_{k\deltat}) - A_\psi(s_{k \deltat}, a_{k \deltat})\\
	\theta^{k+1}_\deltat &= \theta^k_\deltat + 
	\alpha^V \deltat^\beta \frac{\delta Q
	}{\deltat}\partial_\theta V_{\theta^k_\deltat}(s_{k\deltat})\\
	\psi^{k+1}_\deltat &= \psi^k_\deltat + 
	\alpha^A \deltat^\beta \frac{\delta Q
	}{\deltat}
	\partial_\psi A_{\theta^k_\deltat}(s_{k\deltat}, a_{k\deltat})
\end{align}
Under uniform boundedness of the Hessian of $s \mapsto V_\theta(s)$,
one can show
\begin{equation}
	\begin{pmatrix}
		\theta_\deltat^{k+1} \\
		\psi_\deltat^{k+1}
	\end{pmatrix} =
	\begin{pmatrix}
		\theta_\deltat^{k} \\
		\psi_\deltat^{k}
	\end{pmatrix} + \deltat^\beta F(\theta_\deltat^k, \psi_\deltat^k, s_{k \deltat}, a_{k\deltat}) + \bigO(\deltat^\beta \deltat),
\end{equation}
with a $\bigO$ independent of $k$. With the additional hypothesis that the gradient of
$(s, a) \rightarrow \bar{A}_\psi(s, a)$ is uniformly bounded, we have
\begin{itemize}
	\item For $\beta = 1$, a proof scheme identical to that of Thm.~\ref{th:traj-conv} shows that
discrete trajectories converge pointwise to continuous trajectories defined by the differential equation
\begin{equation}
	\frac{d}{dt}\begin{pmatrix}
		\theta_t \\
		\psi_t
	\end{pmatrix} =
	F(\theta_t, \psi_t, s_t, a_t),
\end{equation}
which admits unique solutions for all initial parameters, since $F$ is uniformly lipschitz continuous.
\item Similarly, for $\beta > 1$, the proof scheme of Thm.~\ref{th:traj-conv} shows that
discrete trajectories converge pointwise to continuous trajectories defined by the differential equation
\begin{equation}
	\frac{d}{dt}\begin{pmatrix}
		\theta_t \\
		\psi_t
	\end{pmatrix} = 0
\end{equation}
and thus that trajectories shrink to a single point as $\deltat$ goes to $0$.
\end{itemize}
