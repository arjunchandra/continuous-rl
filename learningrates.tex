%!TEX root = main.tex
\section{An aside on learning rates}
\label{sec:lr}
Defining $P_{ij} = P(j\mid i)$, $\mathrm{TD}(0)$ can be rewritten as
\begin{align*}
	V^{t+1} - V^* &= V^t - V^* \\
		      &+ \alpha E_{s_t}\left[{(E_{s_{t+1}} - P^T E_{s_t})}^T R + 
		\gamma E_{s_{t+1}}^T V^t - \gamma E_{s_t}^T P V^* - E_{s_t}^T(V^t-V^*)
	\right]
\end{align*}
or, with $W^t = V^t - V^*$,
\begin{align*}
	W^{t+1} = \left[I + \alpha E_{s_t}{(\gamma E_{s_{t+1}} - E_{s_t})}^T\right] W^t +
	\alpha E_{s_t}{(E_{s_{t+1}} - P^T E_{s_t})}^T (R + \gamma V^*)
\end{align*}
We would like to find conditions on $\alpha$ that turn $\|W^t\|^2$ into a submartingale. 
Rewrite
\begin{align*}
	W^{t+1} = X^t + Y^{t+1}
\end{align*}
where
\begin{align*}
	X^t &= \left[I - \alpha E_{s_t}E_{s_t}^T\right] W^t - 
	\alpha E_{s_t}E_{s_t}^T P{(R + \gamma V^*)}\\
	Y^{t+1} &= \alpha \gamma E_{s_t}E_{s_{t+1}}^T W^t + \alpha E_{s_t}E_{s_{t+1}}^T(R + \gamma V^*)\\
	&= \alpha E_{s_t}E_{s_{t+1}}^T(R + \gamma V^t).
\end{align*}
Recall that $\Ep{E_{s_{t+1}}} = P^T E_{s_t}$.
\begin{align}
	\Ep{Y^{t+1}} &= \alpha E_{s_t} E_{s_t}^T P (R + \gamma V^t)\\
	\Ep{W^{t+1}} &= \left[I + \alpha E_{s_t}E_{s_t}^T(\gamma P - I)\right] W^t.
\end{align}
Besides,
\begin{align*}
	\Ep{E_{s_{t+1}}E_{s_t}^T E_{s_t} E_{s_{t+1}}^T} &=
	\sum\limits_{s_{t+1}} P(s_{t+1}\mid s_t) E_{s_{t+1}} E_{s_{t+1}}^T\\
	&= \mathrm{diag}{(P^T E_{s_t})}
\end{align*}
yielding
\begin{align}
	\Ep{Y^{t+1}^T Y^{t+1}} &= \alpha ^ 2
	\Ep{{\left(R(s_{t+1}) + \gamma V^t(s_{t+1})\right)}^2}.
\end{align}
Now
\begin{align*}
	\Ep{W^{t+1}^T W^{t+1}} &= X^t^T X^t + 2X^t^T \Ep{Y^{t+1}}
	+ \Ep{Y^{t+1}^TY^{t+1}}\\
	\Ep{W^{t+1}}^T\Ep{W^{t+1}} &= X^t^T X^t + 2X^t^T \Ep{Y^{t+1}}
				 \\&+ \Ep{Y^{t+1}}^T\Ep{Y^{t+1}}.
\end{align*}
Consequently
\begin{align*}
	\Ep{W^{t+1}^T W^{t+1}} &= \Ep{W^{t+1}}^T\Ep{W^{t+1}} + \Ep{Y^{t+1}^T Y^{t+1}}\\
			       &- \Ep{Y^{t+1}}^T\Ep{Y^{t+1}}
\end{align*}
and finally
\begin{align*}
	\Ep{W^{t+1}^T W^{t+1}} = \|\left[I + \alpha E_{s_t} E_{s_t}^T(\gamma P - I)\right]W^t\|^2 + \alpha ^ 2 \mathrm{Var}\left[R(s_{t+1}) + \gamma V^t(s_{t+1})\mid \pfield\right].
\end{align*}
With no further assumptions or approximations, considered learning rates must verify 
stringent conditions for convergence to hold, typically
\begin{align*}
	\sum\limits \alpha_t = +\infty\\
	\sum\limits \alpha_t ^ 2 < + \infty.
\end{align*}
We turn to a simpler dynamic to obtain less stringent conditions on the learning rates,
\begin{align*}
	\Ep{W^{t+1}^T W^{t+1}} = \|\left[I + \alpha \diag{\mu}(\gamma P - I)\right]W^t\|^2 + \alpha ^ 2 \mathrm{Var}\left[R(s_{t+1}) + \gamma V^t(s_{t+1})\mid \pfield\right].
\end{align*}
This dynamic amounts to only perform updates on $V^t$ once you have enough
samples from the markov chain to have a representative sample of its invariant
distribution $\mu$.

Now, defining $R_+ = \max\limits_s |R(s)|$,
\begin{align}
	\Ep{W^{t+1}^T W^{t+1}} \leq \|\left[I + \alpha \diag{\mu}(\gamma P - I)\right]W^t\|^2 + \frac{\alpha^2}{(1 - \gamma)^2} R_+^2.
\end{align}
Define $\lmax = \max(\alpha (\gamma + 1) \max(\mu) - 1, 1 - \alpha (1 - \gamma) \min(\mu))$ and $N^t$ as
\begin{align}
	N^{t+1} = \lmax N_t + \frac{\alpha^2}{(1 - \gamma)^2} R_+^2.
\end{align}
Then $N^t$ upper bounds $\Ep{W^t^T W^t}$ and $N^t \to \frac{\alpha^2 R_+^2} {(1
- \gamma)^2 (1 - \lmax)} = S \varepsilon^2$, with $S$ the number of states,
thus, on average, $V^t$ approximates $V^*$ up to $\varepsilon$ componentwise as
$t\to\infty$.  To obtain a precision up to $\deltat$, you tipically need
\begin{align}
	\alpha = \frac{\sqrt{S}\deltat^2(1 - \gamma)^2\min(2 - (\gamma + 1)\max(\mu),(1-\gamma)\min(\mu))}{R_+^2}
\end{align}
in most cases, $2 - (\gamma + 1)\max(\mu) > (1 - \gamma) \min(\mu)$, yielding
\begin{align}
	\alpha = \frac{\sqrt{S}\deltat^5 (1 - \gamma')^3 \min(\mu)}{R_+^2},
\end{align}
with $\gamma' = (1 - (1 - \gamma) \deltat)$, the discretization insensitive discount.
Now, if the invariant distribution is quasi uniform and the dimension of the state space
is $D$, $\min(\mu) \approx \deltat^D$, and $S \approx \deltat^{-D}$.

Consequently,
\begin{equation}
	\alpha = \frac{\deltat^{5 + \frac{D}{2}}(1 - \gamma')}{R_+^2}.
\end{equation}
